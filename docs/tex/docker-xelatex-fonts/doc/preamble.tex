% !TeX spellcheck = russian-aot-ieyo
% Зачем: Определяет класс документа (То, как будет выглядеть документ)
% Примечание: параметр draft помечает строки, вышедшие за границы страницы, прямоугольником, в фильной версии его нужно удалить.
\documentclass[a4paper,14pt,russian,oneside,final]{extreport}

% Поддержка русского языка (XeLaTeX)
%% Для включения переносов в английском поместите английский текст в секцию \textenglish.
%%    Например, "привет \textenglish{Hello world} еще привет"
\usepackage{polyglossia}   %% загружает пакет многоязыковой вёрстки
\setdefaultlanguage{russian}  %% устанавливает главный язык документа
    %\setdefaultlanguage[babelshorthands=true]{russian}  %% вместо предыдущей строки; доступны команды из пакета babel для русского языка
\setotherlanguage{english} %% объявляет второй язык документа
\defaultfontfeatures{Ligatures={TeX}}  %% свойства шрифтов по умолчанию. Для XeTeX опцию Renderer=Basic можно не указывать, она необходима для LuaTeX
\setmainfont[Ligatures={TeX,Historic}]{Times New Roman} %% задаёт основной шрифт документа
\setromanfont{Times New Roman}
\setsansfont{Arial}
\setmonofont{Courier New}

% Зачем: Убирает ошибку ненахождения Cyrillic в Times New Roman
\newfontfamily\cyrillicfont{Times New Roman} %[Script=Cyrillic]
\newfontfamily{\cyrillicfontrm}{Times New Roman}
\newfontfamily{\cyrillicfonttt}{Courier New}
\newfontfamily{\cyrillicfontsf}{Arial}

% Зачем: Чтобы можно было использовать русские буквы в формулах, но в случае использования предупреждать об этом.
\usepackage[warn]{mathtext}

% Зачем: Добавляет поддержу дополнительных размеров текста 8pt, 9pt, 10pt, 11pt, 12pt, 14pt, 17pt, and 20pt.
% Почему: Пункт 2.1.1 Требований по оформлению пояснительной записки.
\usepackage{extsizes}

% Зачем: Длинна, пимерно соответвующая 5 символам
% Почему: Требования содержат странное требование про отсупы в 5 символов (для немоноширинного шрифта :| )
\newlength{\fivecharsapprox}
\setlength{\fivecharsapprox}{1.25cm}


% Зачем: Добавляет отступы для абзацев.
% Почему: Пункт 2.1.3 Требований по оформлению пояснительной записки.
\usepackage{indentfirst}
\setlength{\parindent}{\fivecharsapprox} % Примерно соответсвует 5 символам.


% Зачем: Настраивает отступы от границ страницы.
% Почему: Пункт 2.1.2 Требований по оформлению пояснительной записки.
\usepackage[left=3cm,top=2.0cm,right=1.5cm,bottom=2.7cm]{geometry}


% Зачем: Настраивает межстрочный интервал, для размещения 40 +/- 3 строки текста на странице.
% Почему: Пункт 2.1.1 Требований по оформлению пояснительной записки.
\usepackage[nodisplayskipstretch]{setspace}
\setstretch{1.0}
%\onehalfspacing

% Зачем: Выбор шрифта по-умолчанию.
% Почему: Пункт 2.1.1 Требований по оформлению пояснительной записки.
% Примечание: В требованиях не указан, какой именно шрифт использовать. По традиции используем TNR.
\renewcommand{\rmdefault}{ftm} % Times New Roman


% Зачем: Отключает использование изменяемых межсловных пробелов.
% Почему: Так не принято делать в текстах на русском языке.
\frenchspacing


% Зачем: Сброс счетчика сносок для каждой страницы
% Примечание: в "Требованиях по оформлению пояснительной записки" не указано, как нужно делать, но в других БГУИРовских докуметах рекомендуется нумерация отдельная для каждой страницы
\usepackage{perpage}
\MakePerPage{footnote}


% Зачем: Добавляет скобку 1) к номеру сноски
% Почему: Пункты 2.9.2 и 2.9.1 Требований по оформлению пояснительной записки.
\makeatletter
\def\@makefnmark{\hbox{\@textsuperscript{\normalfont\@thefnmark)}}}
\makeatother


% Зачем: Расположение сносок внизу страницы
% Почему: Пункт 2.9.2 Требований по оформлению пояснительной записки.
\usepackage[bottom]{footmisc}

% Зачем: Пункты (в терминологии требований) в терминологии TeX subsubsection должны нумероваться
% Почему: Пункт 2.2.3 Требований по оформлению пояснительной записки.
\setcounter{secnumdepth}{3}


% Зачем: Настраивает отступ между таблицей с содержанимем и словом СОДЕРЖАНИЕ
% Почему: Пункт 2.2.7 Требований по оформлению пояснительной записки.
\usepackage{tocloft}
\setlength{\cftbeforetoctitleskip}{-1em}
\setlength{\cftaftertoctitleskip}{1em}

% Зачем: Определяет отступы слева для записей в таблице содержания.
% Почему: Пункт 2.2.7 Требований по оформлению пояснительной записки.
\makeatletter
\renewcommand{\l@section}{\@dottedtocline{1}{0.5em}{1.2em}}
\renewcommand{\l@subsection}{\@dottedtocline{2}{1.7em}{2.0em}}

\makeatother


% Зачем: Работа с колонтитулами
\usepackage{fancyhdr} % пакет для установки колонтитулов
\pagestyle{fancy} % смена стиля оформления страниц


% Зачем: Нумерация страниц располагается справа снизу страницы
% Почему: Пункт 2.2.8 Требований по оформлению пояснительной записки.
\fancyhf{} % очистка текущих значений
\fancyfoot[R]{\thepage} % установка верхнего колонтитула
\renewcommand{\footrulewidth}{0pt} % убрать разделительную линию внизу страницы
\renewcommand{\headrulewidth}{0pt} % убрать разделительную линию вверху страницы
\fancypagestyle{plain}{
    \fancyhf{}
    \rfoot{\thepage}}


% Зачем: Задает стиль заголовков раздела жирным шрифтом, прописными буквами, без точки в конце
% Почему: Пункты 2.1.1, 2.2.5, 2.2.6 и ПРИЛОЖЕНИЕ Л Требований по оформлению пояснительной записки.
\makeatletter
\renewcommand\thesubsection{\thesection.\arabic{subsection}}
\renewcommand\thesection{\arabic{section}}
\makeatother

\makeatletter
\renewcommand\section{%
  \clearpage\@startsection {section}{1}%
    {\fivecharsapprox}%
    {-1em \@plus -1ex \@minus -.2ex}%
    {1em \@plus .2ex}%
    {\raggedright\hyphenpenalty=10000\normalfont\MakeUppercase}}
\makeatother

% Зачем: Задает стиль заголовков подразделов
% Почему: Пункты 2.1.1, 2.2.5 и ПРИЛОЖЕНИЕ Л Требований по оформлению пояснительной записки.
\makeatletter
\renewcommand\subsection{%
  \@startsection{subsection}{2}%
    {\fivecharsapprox}%
    {-1em \@plus -1ex \@minus -.2ex}%
    {1em \@plus .2ex}%
    {\raggedright\hyphenpenalty=10000\normalfont\normalsize}}
\makeatother


% Зачем: Задает стиль заголовков пунктов
% Почему: Пункты 2.1.1, 2.2.5 и ПРИЛОЖЕНИЕ Л Требований по оформлению пояснительной записки.
\makeatletter
\renewcommand\subsubsection{
  \@startsection{subsubsection}{3}%
    {\fivecharsapprox}%
    {-1em \@plus -1ex \@minus -.2ex}%
    {1em \@plus .2ex}%
    {\raggedright\hyphenpenalty=10000\normalfont\normalsize}}
\makeatother

% Зачем: для оформления введения и заключения, они должны быть выровнены по центру.
% Почему: Пункты 1.1.15 и 1.1.11 Требований по оформлению пояснительной записки.
\makeatletter
\newcommand\sectioncentered{%
  \clearpage\@startsection {section}{1}%
    {\z@}%
    {-1em \@plus -1ex \@minus -.2ex}%
    {1em \@plus .2ex}%
    {\centering\hyphenpenalty=10000\normalfont\MakeUppercase}%
    }
\makeatother

% Зачем: отступ от номера до заголовка должен составлять 1 пробел.
% Для Times New Roman это приблизительно 0.5em
\usepackage{titlesec}
\titleformat{\section}
  {}{\textbf{\thesection}}{0.5em}{}
\titleformat{\subsection}
  {}{\textbf{\thesubsection}}{0.5em}{}
\titleformat{\subsubsection}
  {}{\textbf{\thesubsubsection}}{0.5em}{}

% Зачем: Задает стиль библиографии
% Почему: Пункт 2.8.6 Требований по оформлению пояснительной записки.
\bibliographystyle{styles/belarus-specific-utf8gost780u}

% Зачем: форматирование даты
\usepackage[ddmmyy]{datetime}
\renewcommand{\dateseparator}{.}

% Работа со строками
\usepackage{xstring}

% #1 = day
% #2 = month
% #3 = year
\def\printdate#1#2#3{\ddmmyyyydate\formatdate{#1}{#2}{#3}}

% Парсит формат даты "yyyy-mm-dd" и печатает его по ГОСТу
\def\parsedate#1{%
  \StrBefore[1]{#1}{-}[\myYear]%
  \StrBetween[1,2]{#1}{-}{-}[\myMonth]%
  \StrBehind[2]{#1}{-}[\myDay]%
  \printdate{\myDay}{\myMonth}{\myYear}%
}

\def\BibUrl#1{Режим доступа : \url{#1}}
\def\BibDate#1{Дата доступа : \parsedate{#1}}
\def\BibTitleSuffix{ [Электронный ресурс]}
\def\BibTitleFormat#1{\StrSubstitute{#1}{ - }{ -- }}

% Зачем: Пакет для вставки картинок
% Примечание: Объяснение, зачем final - http://tex.stackexchange.com/questions/11004/why-does-the-image-not-appear
\usepackage[final]{graphicx}
\DeclareGraphicsExtensions{.pdf,.png,.jpg,.eps}


% Зачем: Директория в которой будет происходить поиск картинок
\graphicspath{{images/}}

\setlength{\belowcaptionskip}{-10pt}
\setlength{\abovecaptionskip}{0pt}

% Зачем: Добавление подписей к рисункам
\usepackage[nooneline]{caption}
\usepackage{subcaption}

% Зачем: чтобы работала \No в новых латехах
\DeclareRobustCommand{\No}{\ifmmode{\nfss@text{\textnumero}}\else\textnumero\fi}

% Зачем: поворот ячеек таблиц на 90 градусов
\usepackage{rotating}
\DeclareRobustCommand{\povernut}[1]{\begin{sideways}{#1}\end{sideways}}


% Зачем: когда в формулах много кириллических символов команда \text{} занимает много места
\DeclareRobustCommand{\x}[1]{\text{#1}}

% Зачем: Задание подписей, разделителя и нумерации частей рисунков
% Почему: Пункт 2.5.5 Требований по оформлению пояснительной записки.
\DeclareCaptionLabelFormat{stbfigure}{\\Рисунок \arabic{section}.\arabic{figure}}
\DeclareCaptionLabelFormat{stbtable}{Таблица \arabic{section}.\arabic{table}}
\DeclareCaptionLabelSeparator{stb}{~--~}
\captionsetup{labelsep=stb}
\captionsetup[figure]{labelformat=stbfigure,justification=centering}
\captionsetup[table]{labelformat=stbtable,justification=raggedright,aboveskip=0pt}

% Зачем: Окружения для оформления формул
% Почему: Пункт 2.4.7 требований по оформлению пояснительной записки и специфические требования различных кафедр
% Пример использования смотри в course_content.tex, строка 5
\usepackage{calc}
\newlength{\lengthWordWhere}
\settowidth{\lengthWordWhere}{где}
\newenvironment{explanationx}
    {%
    %%% Следующие строки определяют специфические требования разных редакций стандартов. Раскоменнтируйте нужную строку
    %% стандартный абзац, СТП-01 2010
    %\begin{itemize}[leftmargin=0cm, itemindent=\parindent + \lengthWordWhere + \labelsep, labelsep=\labelsep]
    %% без отступа, СТП-01 2013
    \begin{itemize}[leftmargin=0cm, itemindent=\lengthWordWhere + \labelsep , labelsep=\labelsep]%
    \renewcommand\labelitemi{}%
    }
    {%
    %\\[\parsep]
    \end{itemize}
    }

% Старое окружение для "где". Сохранено для совместимости
\usepackage{tabularx}

\newenvironment{explanation}
    {
    %%% Следующие строки определяют специфические требования разных редакций стандартов. Раскоменнтируйте нужные 2 строки
    %% стандартный абзац, СТП-01 2010
    %\par
    %\tabularx{\textwidth-\fivecharsapprox}{@{}ll@{ --- } X }
    %% без отступа, СТП-01 2013
    \noindent
    \tabularx{\textwidth}{@{}ll@{ --- } X }
    }
    {
    \\[\parsep]
    \endtabularx
    }


% Зачем: Удобная верстка многострочных формул, масштабирующийся текст в формулах, формулы в рамках и др
\usepackage{amsmath}


% Зачем: Поддержка ажурного и готического шрифтов
\usepackage{amsfonts}


% Зачем: amsfonts + несколько сотен дополнительных математических символов
\usepackage{amssymb}


% Зачем: Окружения «теорема», «лемма»
\usepackage{amsthm}


% Зачем: Производить арифметические операции во время компиляции TeX файла
\usepackage{calc}

% Зачем: Производить арифметические операции во время компиляции TeX файла
\usepackage{fp}

% Зачем: Пакет для работы с перечислениями
\usepackage{enumitem}
\makeatletter
 \AddEnumerateCounter{\asbuk}{\@asbuk}{щ)}
\makeatother


% Зачем: Устанавливает символ начала простого перечисления
% Почему: Пункт 2.3.5 Требований по оформлению пояснительной записки.
\setlist{nolistsep}


% Зачем: Устанавливает символ начала именованного перечисления
% Почему: Пункт 2.3.8 Требований по оформлению пояснительной записки.
\renewcommand{\labelenumi}{\asbuk{enumi})}
\renewcommand{\labelenumii}{\arabic{enumii})}

% Зачем: Устанавливает отступ от границы документа до символа списка, чтобы этот отступ равнялся отступу параграфа
% Почему: Пункт 2.3.5 Требований по оформлению пояснительной записки.

\setlist[itemize,0]{itemindent=\parindent + 2.2ex,leftmargin=0ex,label=--}
\setlist[enumerate,1]{itemindent=\parindent + 2.7ex,leftmargin=0ex}
\setlist[enumerate,2]{itemindent=\parindent + \parindent - 2.7ex}

% Зачем: добавление возможности создания нумерованного списка
\newenvironment{enumerate_num}
  {\begin{enumerate}[label=\arabic*]}
  {\end{enumerate}}

% Зачем: добавление возможности создания вложеннного нумерованного списка с  "жирными" цифрами
\newlength{\taskLiskSubsectionDelta}
\setlength{\taskLiskSubsectionDelta}{0.75cm}
\newlength{\taskLiskSectionItemLeftIndent}
\setlength{\taskLiskSectionItemLeftIndent}{0.5cm}
\newenvironment{enumerate_nest_bold_num}
  {\begin{enumerate}[label*=\textbf{.\arabic*},
    leftmargin=*, itemindent=0cm]}
  {\end{enumerate}}

% Зачем: для основного списка из листа задания возможность добавления встроенных элементов без отступов перед ними
\newenvironment{enumerate_for_empty}
  {\begin{enumerate}[leftmargin=0cm, itemindent=0cm]}
  {\end{enumerate}}

% Зачем: Включение номера раздела в номер формулы. Нумерация формул внутри раздела.
\AtBeginDocument{\numberwithin{equation}{section}}

% Зачем: Включение номера раздела в номер таблицы. Нумерация таблиц внутри раздела.
\AtBeginDocument{\numberwithin{table}{section}}

% Зачем: Включение номера раздела в номер рисунка. Нумерация рисунков внутри раздела.
\AtBeginDocument{\numberwithin{figure}{section}}

\let\oldequation=\equation
\let\endoldequation=\endequation
\renewenvironment{equation}{\vspace{-0.3em}\begin{oldequation}}{\end{oldequation}\vspace{-0.9em}}

% Зачем: Дополнительные возможности в форматировании таблиц
\usepackage{makecell}
\usepackage{multirow}
\usepackage{array}


% Зачем: "Умная" запятая в математических формулах. В дробных числах не добавляет пробел
% Почему: В требованиях не нашел, но в русском языке для дробных чисел используется {,} а не {.}
\usepackage{icomma}

% Зачем: макрос для печати римских чисел
\makeatletter
\newcommand{\rmnum}[1]{\romannumeral #1}
\newcommand{\Rmnum}[1]{\expandafter\@slowromancap\romannumeral #1@}
\makeatother


% Зачем: Управление выводом чисел.
\usepackage{sistyle}
\SIdecimalsign{,}

% Зачем: inline-коментирование содержимого.
\newcommand{\ignore}[2]{\hspace{0in}#2}


% Зачем: Возможность коментировать большие участки документа
\usepackage{verbatim}


\usepackage{xcolor}


% Зачем: Оформление листингов кода
% Примечание: final нужен для переопределения режима draft, в котором листинги не выводятся в документ.
\usepackage[final]{listings}

\renewcommand{\lstlistingname}{Листинг}

% Зачем: Нумерация листингов в пределах секции
\AtBeginDocument{\numberwithin{lstlisting}{section}}

\usepackage[normalem]{ulem}

\usepackage[final,hidelinks]{hyperref}
% Делаем все \url с обычным шрифтом
\renewcommand{\UrlFont}{}
% \renewcommand{\UrlFont}{\small\rmfamily\tt}

\usepackage[square,numbers,sort&compress]{natbib}
\setlength{\bibsep}{0em}

% Магия для подсчета разнообразных объектов в документе
\usepackage{lastpage}
\usepackage{totcount}
\regtotcounter{section}

\usepackage{etoolbox}

\newcounter{totfigures}
\newcounter{tottables}
\newcounter{totreferences}
\newcounter{totequation}

\providecommand\totfig{}
\providecommand\tottab{}
\providecommand\totref{}
\providecommand\toteq{}

\makeatletter
\AtEndDocument{%
  \addtocounter{totfigures}{\value{figure}}%
  \addtocounter{tottables}{\value{table}}%
  \addtocounter{totequation}{\value{equation}}
  \immediate\write\@mainaux{%
    \string\gdef\string\totfig{\number\value{totfigures}}%
    \string\gdef\string\tottab{\number\value{tottables}}%
    \string\gdef\string\totref{\number\value{totreferences}}%
    \string\gdef\string\toteq{\number\value{totequation}}%
  }%
}
\makeatother

\pretocmd{\section}{\addtocounter{totfigures}{\value{figure}}\setcounter{figure}{0}}{}{}
\pretocmd{\section}{\addtocounter{tottables}{\value{table}}\setcounter{table}{0}}{}{}
\pretocmd{\section}{\addtocounter{totequation}{\value{equation}}\setcounter{equation}{0}}{}{}
\pretocmd{\bibitem}{\addtocounter{totreferences}{1}}{}{}



% Для оформления таблиц не влязящих на 1 страницу
\usepackage{longtable}

% Для включения pdf документов в результирующий файл
\usepackage{pdfpages}

% Для использования знака градуса и других знаков
% http://ctan.org/pkg/gensymb
\usepackage{gensymb}

% Зачем: преобразовывать текст в верхний регистр командой MakeTextUppercase
\usepackage{textcase}

% Зачем: Переносы в словах с тире.
% Тире в словае заменяем на \hyph: аппаратно\hyphпрограммный.
% https://stackoverflow.com/questions/2193307/how-to-get-latex-to-hyphenate-a-word-that-contains-a-dash#
\def\hyph{-\penalty0\hskip0pt\relax}

% Добавляем абзацный отступ для библиографии
% https://github.com/mstyura/bsuir-diploma-latex/issues/19
\setlength\bibindent{\fivecharsapprox}

\makeatletter
% Зачем: устанавливаем, как оформлять индекс
% Зачем: в списке литературы выровнять индексы по левому краю (\hfill)
\renewcommand\@biblabel[1]{[#1]\hspace*{\fill}}
\makeatother

% \z@ == 0cm
\makeatletter
\renewcommand\NAT@bibsetnum[1]{%
  % alignment, based on first? element, not on widest
  % \settowidth\labelwidth{\@biblabel{#1}}%
  \setlength{\leftmargin}{0cm}%
  \setlength{\itemindent}{\bibindent+\labelwidth}%
  % \setlength{\itemindent}{\labelwidth}%
  \addtolength{\labelsep}{-0.5em}%
  \setlength{\itemsep}{\bibsep}\setlength{\parsep}{\z@}%
  % not tested for openbib
   \ifNAT@openbib
     \addtolength{\leftmargin}{\bibindent}%
     \setlength{\itemindent}{-\bibindent}%
     \setlength{\listparindent}{\itemindent}%
     \setlength{\parsep}{10pt}%
   \fi
}
\makeatother

\lstdefinestyle{cstyle}{
   xleftmargin=1.25cm,
   basicstyle=\fontencoding{T1}\small\ttfamily,
   breaklines=true,
   tabsize=8
}
\let\OldTexttt\texttt
\renewcommand{\texttt}[1]{\OldTexttt{\fontencoding{T1}\ttfamily{#1}}}

% Зачем: убрать выход текста за границы страницы при переносе за счет увеличения ширины пробелов (тут в 2 раза)
% \newdimen\origiwstr
% \origiwstr=\fontdimen3\font
% \fontdimen3\font=2\origiwstr
\fontdimen3\font=2\fontdimen3\font

% Зачем: для генерации автоматических окончаний
%% Подключен выше
% \usepackage{xstring}

% Зачем: автоматические окончания
%% #1 значение счетчика строкой
%% #2 окончание для остатка 1 по модулю 10
%% #3 окончание для остатка 2-4 по модулю 10
%% #4 окончание для остатка 5-9 и 0 по модулю 10
\newcommand{\insertNumSomethingText}[4]{%
#1 % вывод количества (числом) + пробел
\StrRight{#1}{1}[\lastsymbol]% Получаем последнюю цифру
\IfStrEqCase{\lastsymbol}{% вывод окончания
  {0}{#4}%
  {1}{#2}%
  {2}{#3}%
  {3}{#3}%
  {4}{#3}%
  {5}{#4}%
  {6}{#4}%
  {7}{#4}%
  {8}{#4}%
  {9}{#4}%
}%
[{Unsuppotred number format}]%Default string value
}

% Зачем: вывод количества рисунков с окончанием
\newcommand{\insertNumFiguresText}{%
\insertNumSomethingText{\totfig}{рисунок}{рисунка}{рисунков}%
}

% Зачем: вывод количества таблиц с окончанием
\newcommand{\insertNumTablesText}{%
\insertNumSomethingText{\tottab}{таблица}{таблицы}{таблиц}%
}

\usepackage{refcount}

% Зачем: вывод количества страниц документа с окончанием
\newcommand{\insertNumPagesText}{%
\insertNumSomethingText{\getpagerefnumber{LastPage}}{страница}{страницы}{страниц}%
}

% Зачем: счетчик для библиографии (литературных источников)
\usepackage{totcount}
\newtotcounter{citenum}
\def\oldcite{}
\let\oldcite=\bibcite
\def\bibcite{\stepcounter{citenum}\oldcite}

% Зачем: вывод количества элементов библиографии с окончанием
\newcommand{\insertNumBibElementsText}{%
\insertNumSomethingText{\total{citenum}}{литературный источник}{литературных источника}{литературных источников}%
}

% Зачем: допустимые буквы для приложений [2.7.2]
\makeatletter
\def\@annexAlph#1{%
  \ifcase#1\or А\or Б\or В\or Г\or Д\or Е\or Ж\or И\or К\or Л\or%
  М\or Н\or П\or Р\or С\or Т\or У\or Ф\or Х\or Ц\or Ш\or Щ\or Э\or%
  Ю\or Я\else\@ctrerr\fi%
}
\def\annexAlph#1{\expandafter\@annexAlph\csname c@#1\endcsname}
\makeatother

% Зачем: счетчик для приложений
\newtotcounter{equation_annex}

% Зачем: создание заголовка приложения
% Параметры:
%% #1 - Заголовок приложения
\newcommand{\annex}[1]{
  \newpage
  \stepcounter{equation_annex}
  \centerline{ПРИЛОЖЕНИЕ \annexAlph{equation_annex}}
  \centerline{(\it{обязательное})}
  \vspace{\baselineskip}
  \centerline{#1}
  \addcontentsline{toc}{section}{ПРИЛОЖЕНИЕ \annexAlph{equation_annex}}
}

% Зачем: вывод количества приложений с окончанием
\newcommand{\insertNumAnnexesText}{%
\insertNumSomethingText{\total{equation_annex}}{приложение}{приложения}{приложений}%
}

% Зачем: список условных сокращений
\usepackage[russian]{nomencl}
\makenomenclature

% Установка заголовка
\renewcommand{\nomname}{УСЛОВНЫЕ СОКРАЩЕНИЯ}

% установка расстояния между элементами в номенклатуре = 0 (из документации)
\setlength{\nomitemsep}{-\parsep}

% установка текста для сокращения в номенклатуре
\renewcommand{\nomlabel}[1]{#1 -- }

% Зачем: установка левого отступа в номенклатуре
\patchcmd{\thenomenclature}
  {\leftmargin\labelwidth}
  {\leftmargin\labelwidth\itemindent\fivecharsapprox }
  {}{}

% Зачем: зануление отступа после сокращения
\patchcmd{\thenomenclature}
  {\advance\leftmargin\labelsep}
  {\labelsep=0cm \advance\leftmargin\labelsep}
  {}{}

% Зачем: обновление форматирования заголовка для номенклатуры
\makeatletter
\patchcmd{\thenomenclature}
  {\tocbasic@listhead{\list@fname}}
  {\sectioncentered*{\list@fname}}
  {}{}
\makeatother

% Зачем: убираем отступ в 1см слева во второй и следующих строках в номенклатуре (отступ label)
\setlength{\nomlabelwidth}{0pt}

% Убирает ошибки для nomenctl о ненахождении соответствующих команд
% Это проблема совместимости русской polyglossia и nomencl
% Подробнее:
%% https://tex.stackexchange.com/questions/449877/xelatex-scrartcl-setmainlanguagerussian-latex-error-chapterformat-und/449880#449880
%% https://github.com/reutenauer/polyglossia/issues/210
\providecommand{\chapterformat}{}
\providecommand{\sectionformat}{}
\providecommand{\subsectionformat}{}
\providecommand{\subsubsectionformat}{}
\providecommand{\paragraphformat}{}
\providecommand{\subparagraphformat}{}
