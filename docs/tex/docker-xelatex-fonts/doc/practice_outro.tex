\sectioncentered*{ЗАКЛЮЧЕНИЕ}
\addcontentsline{toc}{section}{ЗАКЛЮЧЕНИЕ}
\label{sec:outro}

В рамках данного дипломного проекта был разработан программный продукт для анализа данных с помощью произведения распределённых вычислений.
Данный продукт может использоваться в двух режимах работы:
\begin{itemize}
    \item локальный;
    \item распределённый.
\end{itemize}

При работе в локальном режиме, все компоненты системы разворачиваются на используемой машине, и позволяют производить вычисления посредством контейнеризации.
Для получения данных использовался внешний сервис OpenWeatherMap, который предоставляет доступ к индексам загрязнений окружающей среды.
Собранные данные послу обработки кэшируются и отправляются в используемый брокер сообщений.

Одновременно с этим, запускается интерактивный документ Zeppelin, который обрабатывает сообщения, принятые за конкретный промежуток времени.
После обработки происходит регрессионный анализ обработанных данных и в итоге получаются предполагаемые индексы загрязнения окружающей среды на несколько лет вперёд для определённого географического положения.

Данный продукт может быть использован в качестве получения и обработки других данных в режиме реального времени.
Наиболее яркий пример, это анализ произведённых транзакций в режиме реального времени.

Дипломный проект был разработан в полном объёме и выполняет возложенные на него функции.
В дальнейшем он может быть улучшен следующим образом:
\begin{itemize}
    \item поддержка других брокеров сообщений;
    \item поддержка Microsoft Azure Blob Storage;
    \item увеличение количества источников данных;
    \item увеличение количества обработчиков данных.
\end{itemize}

Таким образом, разработанный продукт может применяться как самостоятельный проект, либо в качестве фреймворка для распределённого анализа данных.
Гибкая архитектура позволяет использовать данный проект для любого формата данных.

