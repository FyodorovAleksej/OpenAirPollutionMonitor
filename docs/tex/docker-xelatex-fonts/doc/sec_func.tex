\section{ФУНКЦИОНАЛЬНОЕ ПРОЕКТИРОВАНИЕ}
\label{sec:func}

Рассмотрим подробно функционирование программы.
Для этого проведем анализ основных модулей программы и рассмотрим их зависимости.
А также проанализируем все модули, которые входят в состав кода программы, и рассмотрим назначение всех методов и переменных этих модулей.
В разрабатываемом приложении можно выделить следующие модули:
\begin{itemize}
    \item модуль конфигурации загрузчика;
    \item модуль хранения загрузчика;
    \item модуль внешнего подключения загрузчика;
    \item модуль обработки загрузчика;
    \item модуль отправки загрузчика;
    \item модуль приёма обработчика;
    \item модуль анализа обработчика;
    \item модуль сохранения обработчика;
    \item модуль визуализации обработчика.
\end{itemize}

Изначально пользователю предоставляется интерактивный документ на Apache Zepelin, в котором он может просмотреть обработанные данные.

\subsection{Классы модуля конфигурации загрузчика}
Для конфигурации приложения используются файлы конфигурации.
Классы данного модуля как раз представляют объекты конфигураций, которые создаются на основе файлов конфигураций.
Главным объектом конфигурации является класс \texttt{ApplicationConfig}.
Он содержит в себе все остальные конфигурации, необходимые для работы всего приложения.
Сам объект этого класса строится с помощью метода \texttt{parse\_from\_file}.
Этот класс включает в себя более специфические конфигурации, представленные классами: \texttt{APIConfig}, \texttt{FSConfig}, \texttt{KafkaConfig}.

\subsubsection{Класс APIConfig}

Данный класс представляет из себя конфигурацию для работы с сервисом OpenWeatherMap.
Сам объект класса создаётся из списка строк, которые представляют собой строки файла конфигурации, ответственные за настройку API.
Конфигурация содержит в себе следующие поля:
\begin{itemize}
    \item \texttt{api\_key};
    \item \texttt{host}.
\end{itemize}

Для использования сервиса OpenWeatherMap необходимо предоставить ключ, который выдаётся при оформлении подписки на определённый сервис API.
Данный параметр как раз и является этим ключом.
Этот ключ передаётся в качестве параметра при использовании GET запроса к сервису.
Как раз для этого ключа и используется параметр \texttt{api\_key} в конфигурации.

Параметр \texttt{host} устанавливает корневой путь к сервису OpenWeatherMap.
Относительно этого пути и строятся все HTTP запросы для получения данных.

Для создания экземпляра рассматриваемого класса, необходимо вызвать статический метод \texttt{parse\_from\_lines}, который принимает список строк, из которых и будет строиться конфигурация.
Для получения ключа для использования API и адреса сервиса OpenWeatherMap из конфигурации, необходимо воспользоваться методами \texttt{get\_api\_key} и \texttt{get\_host} соответственно.

\subsubsection{Класс FSConfig}

Так как для всего приложения используется полноценный кластер, то и необходима возможность использования приложения в экосистеме Hadoop.
В этой экосистеме используется HDFS, как уже неоднократно отмечалось.
Если для работы с локальной файловой системой можно обойтись без каких-либо конфигураций, то для HDFS необходимо знать её корневой адрес.
Данный адрес и является полем \texttt{host} в рассматриваемой конфигурации и задаётся с помощью файлов конфигурации.

Для создания экземпляра этого класса, необходимо вызвать статический метод \texttt{parse\_from\_lines}, который принимает список строк, из которых и будет строиться конфигурация.
Для получения адреса к корневому каталогу в используемой локальной системе из конфигурации, необходимо воспользоваться методами \texttt{get\_host}.

\subsubsection{Класс KafkaProducerConfig}

Данный класс представляет из себя конфигурацию для работы с платформой kafka.
Сам объект класса создаётся из списка строк, которые представляют собой строки файла конфигурации, ответственные за настройку kafka producer.
Конфигурация содержит в себе следующие поля:
\begin{itemize}
    \item \texttt{bootstrap\_servers};
    \item \texttt{client\_id};
    \item \texttt{key\_serializer};
    \item \texttt{value\_serializer};
    \item \texttt{acks};
    \item \texttt{compression\_type};
    \item \texttt{retries};
    \item \texttt{batch\_size};
    \item \texttt{max\_request\_size};
    \item \texttt{request\_timeout\_ms};
    \item \texttt{security\_protocol}.
\end{itemize}

Параметр \texttt{bootstrap\_servers} указывает адреса kafka серверов в формате <<host[:port]>>.

Параметр \texttt{client\_id} указывает уникальный идентификатор клиента.
Данный идентификатор используется для логирования.

Параметр \texttt{key\_serializer} указывает имя класса сериализатора для ключа.
Данный параметр необходим для возможности передачи пользовательских объектов в качестве ключа.

Аналогичным является параметр \texttt{value\_serializer}, который указывает имя класса сериализатора для значения.
Этот параметр необходим для возможности передачи пользовательских объектов в качестве значения.
Также kafka предоставляет набор стандартных классов для сериализации и десериализации.
Например \texttt{org.apache.kafka.common.serialization.StringSerializer} и \texttt{org.apache.kafka.common.serialization.StringDeserializer}.

Параметр \texttt{acks} указывает требуемое количество подтверждений от kafka серверов.
Возможные значения:
\begin{itemize}
    \item \texttt{0}. Означает, что подтверждения не требуются;
    \item \texttt{1}. Используется по умолчанию. Означает, что требуется хотя бы одно подтверждение;
    \item \texttt{all}. Означает, что для каждого запроса необходимо подтверждение.
\end{itemize}

Параметр \texttt{compression\_type} указывает тип сжатия для всех отправленных данных, либо его отсутствие.
Возможные значения:
\begin{itemize}
    \item \texttt{gzip};
    \item \texttt{snappy};
    \item \texttt{lz4};
    \item \texttt{None}. Используется по умолчанию.
\end{itemize}

Параметр \texttt{retries} указывает количество попыток отправить данные заново при неудаче.

Параметр \texttt{batch\_size} указывает максимальный размер пакета для отправки.
При превышении данного размера, данные будут разбиты на несколько пакетов.

Параметр \texttt{request\_timeout\_ms} указывает максимальное время ожидания ответа при отправке пакета в миллисекундах.
При превышении данного времени - считается, что пакет не был доставлен.

Параметр \texttt{security\_protocol} указывает используемый протокол для связи с серверами kafka.
Возможные значения:
\begin{itemize}
    \item \texttt{PLAINTEXT}. Используется по умолчанию;
    \item \texttt{SSL};
    \item \texttt{SASL\_PLAINTEXT};
    \item \texttt{SASL\_SSL}.
\end{itemize}

Для того, чтобы создать экземпляр класса, следует использовать статический метод \texttt{parse\_from\_lines}, который принимает список строк с конфигурацией kafka consumer из файлов конфигурации.

\subsection{Классы модуля хранения загрузчика}
Для сохранения данных может использоваться как локальная файловая система, так и распределённая (HDFS).
Для обеспечения работы с файловой системой, необходимо реализовать базовые операции, которые указаны в интерфейсе \texttt{FileSystemAdapter}.
Его реализация \texttt{DefaultFileSystem} предназначена для произведения операций с помощью локальной файловой системы.
А реализация \texttt{DistributedFileSystem} предназначена для произведения операция с помощью распределённой файловой системы (HDFS).

\subsubsection{Интерфейс FileSystemAdapter}
Предназначен для осуществления операций с файловой системой.
Содержит следующие методы:
\begin{itemize}
    \item \texttt{write\_file}
    \item \texttt{append\_to\_file}
    \item \texttt{read\_file}
    \item \texttt{remove\_file}
    \item \texttt{mkdir}
    \item \texttt{ls}
    \item \texttt{is\_exist}
    \item \texttt{to\_file\_path}
\end{itemize}

Метод \texttt{write\_file} принимает относительный путь в файловой системе и массив байтов.
Предназначен для записи переданных байтов в файл.
Если переданный файл не существует, то он создаётся.
Если переданный файл уже существует, то он будет перезаписан.
В результате ничего не возвращает.

Метод \texttt{append\_to\_file} принимает относительный путь в файловой системе и массив байтов.
Предназначен для добавления байтов в файл.
Если переданный файл не существует, то он будет создан
В результате ничего не возвращает.

Метод \texttt{read\_file} принимает относительный путь в файловой системе.
Предназначен для чтения байтов из файла.
Если файл не существует, то возвращает \texttt{null}.
В результате возвращает массив считанных байт из файла.

Метод \texttt{remove\_file} принимает относительный путь в файловой системе.
Предназначен для удаления файла.
В результате возвращает \texttt{true} в случае успеха и \texttt{false} если файл не существовал.

Метод \texttt{mkdir} принимает относительный путь в файловой системе.
Предназначен для создания директории. 
В результате ничего не возвращает.

Метод \texttt{ls} принимает относительный путь в файловой системе.
Предназначен для получения всех существующих файлов в директории.
В результате возвращает список файлов, которые находятся в переданной директории.
В случае, если такой директории не существует, то возвращает \texttt{null}.

Метод \texttt{is\_exist} принимает относительный путь в файловой системе.
Предназначен для проверки существования пути в файловой системе.
В результате возвращает \texttt{true}, если путь существует и \texttt{false}, если путь не существует.

Метод \texttt{to\_file\_path} принимает относительный путь в файловой системе к директории, географическую широту, географическую долготу, год.
Предназначен для получения пути к файлу, который содержит результат запроса с переданными параметрами.
В результате возвращает путь к файлу.


\subsubsection{Класс DefaultFileSystem}
Предназначен для осуществления операций с локальной файловой системой.
Имплементирует следующие методы:
\begin{itemize}
    \item \texttt{write\_file}
    \item \texttt{append\_to\_file}
    \item \texttt{read\_file}
    \item \texttt{remove\_file}
    \item \texttt{mkdir}
    \item \texttt{ls}
    \item \texttt{is\_exist}
    \item \texttt{to\_file\_path}
\end{itemize}

Метод \texttt{write\_file} принимает относительный путь в локальной файловой системе и массив байтов.
Предназначен для записи переданных байтов в файл.
Если переданный файл не существует, то он создаётся.
Если переданный файл уже существует, то он будет перезаписан.
В результате ничего не возвращает.

Метод \texttt{append\_to\_file} принимает относительный путь в локальной файловой системе и массив байтов.
Предназначен для добавления байтов в файл.
Если переданный файл не существует, то он будет создан
В результате ничего не возвращает.

Метод \texttt{read\_file} принимает относительный путь в локальной файловой системе.
Предназначен для чтения байтов из файла.
Если файл не существует, то возвращает \texttt{null}.
В результате возвращает массив считанных байт из файла.

Метод \texttt{remove\_file} принимает относительный путь в локальной файловой системе.
Предназначен для удаления файла.
В результате возвращает \texttt{true} в случае успеха и \texttt{false} если файл не существовал.

Метод \texttt{mkdir} принимает относительный путь в локальной файловой системе.
Предназначен для создания директории. 
В результате ничего не возвращает.

Метод \texttt{ls} принимает относительный путь в локальной файловой системе.
Предназначен для получения всех существующих файлов в директории.
В результате возвращает список файлов, которые находятся в переданной директории.
В случае, если такой директории не существует, то возвращает \texttt{null}.

Метод \texttt{is\_exist} принимает относительный путь в локальной файловой системе.
Предназначен для проверки существования пути в локальной файловой системе.
В результате возвращает \texttt{true}, если путь существует и \texttt{false}, если путь не существует.

Метод \texttt{to\_file\_path} принимает относительный путь в локальной файловой системе к директории, географическую широту, географическую долготу, год.
Предназначен для получения пути к файлу, который содержит результат запроса с переданными параметрами.
В результате возвращает путь к файлу.


\subsubsection{Класс DistributedFileSystem}
Предназначен для осуществления операций с распределённой файловой системой (HDFS).
Имплементирует следующие методы:
\begin{itemize}
    \item \texttt{write\_file}
    \item \texttt{append\_to\_file}
    \item \texttt{read\_file}
    \item \texttt{remove\_file}
    \item \texttt{mkdir}
    \item \texttt{ls}
    \item \texttt{is\_exist}
    \item \texttt{to\_file\_path}
\end{itemize}

Метод \texttt{write\_file} принимает относительный путь в распределённой файловой системе (HDFS) и массив байтов.
Предназначен для записи переданных байтов в файл.
Если переданный файл не существует, то он создаётся.
Если переданный файл уже существует, то он будет перезаписан.
В результате ничего не возвращает.

Метод \texttt{append\_to\_file} принимает относительный путь в распределённой файловой системе (HDFS) и массив байтов.
Предназначен для добавления байтов в файл.
Если переданный файл не существует, то он будет создан
В результате ничего не возвращает.

Метод \texttt{read\_file} принимает относительный путь в распределённой файловой системе (HDFS).
Предназначен для чтения байтов из файла.
Если файл не существует, то возвращает \texttt{null}.
В результате возвращает массив считанных байт из файла.

Метод \texttt{remove\_file} принимает относительный путь в распределённой файловой системе (HDFS).
Предназначен для удаления файла.
В результате возвращает \texttt{true} в случае успеха и \texttt{false} если файл не существовал.

Метод \texttt{mkdir} принимает относительный путь в распределённой файловой системе (HDFS).
Предназначен для создания директории. 
В результате ничего не возвращает.

Метод \texttt{ls} принимает относительный путь в распределённой файловой системе (HDFS).
Предназначен для получения всех существующих файлов в директории.
В результате возвращает список файлов, которые находятся в переданной директории.
В случае, если такой директории не существует, то возвращает \texttt{null}.

Метод \texttt{is\_exist} принимает относительный путь в распределённой файловой системе (HDFS).
Предназначен для проверки существования пути в распределённой файловой системе.
В результате возвращает \texttt{true}, если путь существует и \texttt{false}, если путь не существует.

Метод \texttt{to\_file\_path} принимает относительный путь в распределённой файловой системе к директории, географическую широту, географическую долготу, год.
Предназначен для получения пути к файлу, который содержит результат запроса с переданными параметрами.
В результате возвращает путь к файлу.


В качестве аргумента в конструкторе принимает строку с адресом корневого каталога в HDFS.

