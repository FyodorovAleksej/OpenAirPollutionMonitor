\sectioncentered*{ВВЕДЕНИЕ}
\addcontentsline{toc}{section}{ВВЕДЕНИЕ}
\label{sec:intro}

С каждым днём промышленность развивается и одним из самых острых вопросов для человечества остаётся вопрос экологии.
Именно по этой причине возникает необходимость в непрерывном наблюдении и анализе изменений индексов загрязнений.
Для осуществления такого наблюдения необходима система, которая сможет обрабатывать данные в режиме реального времени.

Однако для такой системы будет характерно большое количество данных и большие вычислительные затраты.
Именно по этой причине, зарабатываемый проект будет производить вычисления с помощью распределённой системы.
Распределенная система - такая система, которая работает сразу на нескольких машинах, образующих кластер.
В свою очередь кластер - это набор компьютеров или серверов, объединенных сетью и взаимодействующих между собой.
Важнейшие плюсы такого подхода – это высокодоступность и отказоустойчивость.

Для обеспечивания надёжности хранимых данных также необходима отказоустойчивая система. 
Отказоустойчивая система - это такая система, в которой отсутствует единая точка отказа.
Такую систему можно подстраивать под случающиеся отказы.
Например, при отказе одной машины, остальные смогут продолжить свою работу, так что в целом кластер не отключится.
В качестве машин в кластере могут выступать не только физические, но и виртуальные машины.

Целью данного дипломного проекта является разработка системы, которая позволит осуществлять мониторинг изменений индекса загрязнений окружающей среды.
В качестве источника данных был выбран сервис OpenWeatherMap, который предоставляет API для получения данных загрязнений по CO, NO, SO и OZ.
Однако данных от этого источника недостаточно для тренировки нейронной сети, поэтому для её тренировки был выбран отдельный датасет.

% ОПИСАНИЕ ДАТАСЕТА

Для достижения поставленной цели нужно выполнить следующие задачи:
\begin{itemize}
    \item разработать модуль сбора данных с сервиса OpenWeatherMap;
    \item разработать модуль обработки предоставляемых данных в режиме реального времени;
    \item разработать математическую модель, с помощью которой можно предсказывать дальнейшее изменения значения на основе предыдущих;
    \item разработать систему представления обработанных данных;
\end{itemize}

Дипломный проект будет реализовать следующие функции:
\begin{itemize}
    \item сохранение полученных данных;
    \item представление статистики о полученных данных;
    \item представление ожидаемых предстоящих значений;
\end{itemize}


% ОСТАЛЬНОЕ

% На сегодняшний день встраиваемые системы являются неотъемлемой частью нашей жизни.
% Их используют практически повсеместно -- начиная от автомобилей 
% (системы климат-контроля, автопилотирования)
% и заканчивая холодильниками, стиральными машинами, телевизорами
% и другой бытовой техникой, которыми можно управлять, например,
% с помощью встраиваемых решений или смартфонов.

% Встраиваемая система (англ. embedded system) -- это система, которая выполняет
% определенный производителем ограниченный набор функций,
% который не предназначен для изменения конечным пользователем ~\cite{esd_book}.
% К данным системам обычно устанавливаются требования высокой отказоустойчивости,
% обработки данных в режиме реального времени, практически мгновенной реакции
% на входные воздействия. Из-за этого разработчики
% вынуждены использовать специализированные операционные системы,
% которые соответствуют вышеупомянутым требованиям. Их
% называют операционными системами реального времени
% (англ. RTOS), примерами которых могут служить
% VxWorks, FreeRTOS, TI-RTOS, eCos, RTLinux.

% TODO: RTOS - real-time operating system - в список условных сокращений

% После описания логики работы встраиваемой системы на каком-либо
% языке программирования необходимо проверить описание на соответствие
% вышеописанным требованиям, в том числе и требованиям времени.
% Для этого приходится либо использовать какие-либо готовое решения,
% либо создавать и применять собственный профилировщик задач.
% При использовании какой-либо операционной
% системы необходима поддержка профилирования самой операционной системой,
% иначе может потребоваться правка ее исходных кодов либо добавление
% в каждую задачу кода для сбора статистики по профилированию (что, практически, сопоставимо с
% имлементацией собственной системы профилирования).

% TODO: (в операционных системах реального времени задачами обычно называют потоки либо псевдо-потоки, которые выполняют ту или иную функцию) - в обзор литературы

% Целью данного дипломного проекта является создание модуля сбора статистики
% в режиме реального времени для операционной системы TI-RTOS, который позволит
% производить отладку в условиях, наиболее приближенных к реальным.
% Работа данного проекта не будет требовать наличие подключенного отладчика
% или программатора (например, по интерфейсу JTAG).

% TODO: HWI - hardware interrupts - в список условных сокращений
% TODO: SWI - software interrupts - в список условных сокращений

% Для достижения поставленной цели нужно выполнить следующие задачи:
% \begin{itemize}
%     \item запустить пример работы с операционной системой TI-RTOS, содержащий несколько псевдозадач, проверить корректность его работы (данный пример в дальнейшем будет использоваться в качестве примера для тестирования);
%     \item добавить в пример несколько тестовых HWI и SWI для более полной проверки модуля;
%     \item исследовать возможности профилирования, предоставляемые операционной системой;
%     \item разработать модуль сбора статистики для каждого ядра процессора на основе возможностей, предоставленных операционной системой;
%     \item исследовать возможности межъядерного взаимодействия;
%     % \item разработать модуль сбора статистики со всех ядер;
%     % \item разработать модуль локального сбора сообщений;
%     % \item разработать модуль сбора сообщений со всех ядер;
%     \item разработать модуль для отправки собранной информации для ее внешней обработки и отображения.
% \end{itemize}

% Дипломный проект будет реализовать следующие функции:
% \begin{itemize}
%     \item сбор статистики по задачам;
%     \item сбор статистики по прерывания (HWI и SWI);
%     \item сбор системных и пользовательских сообщений;
%     \item частичное конфигурирование системы профилирования во время ее работы;
%     \item получение данных из системы профилирования для их дальнейшего отображения;
%     \item отображение полученных данных на веб-странице в режиме реального времени.
% \end{itemize}

% Таким образом, можно утверждать, что тема актуальна для реализации
% и может быть в дальнейшем использована для получения коммерческой выгоды.

% 02 марта 2019
% http://systemsauto.ru/heating/climate_control.html - система климат-контроля
% https://auto.vesti.ru/news/show/news_id/701472/ - автопилот спас жизнь
% https://www.dgl.ru/reviews/obzor-umnoy-stiralnoy-mashiny-lg-f12u1hbs4-stiraem-so-smartfona_6854.html - умная стиральная машина
% https://wylsa.com/samsung-family-hub/ - умный холодильник
% https://tech.onliner.by/2018/05/29/yandex-5 - яндекс.станция - новость
% https://station.yandex.ru/#station - яндекс.станция - оф. сайт
% https://yandex-station.ru/analogi-yandeks.stanczii.html - аналоги яндекс станции
% https://the-eye.eu/public/Books/IT%20Various/making_embedded_systems.pdf - making embedded systems (book) - ISBN 978-1-449-30214-6
% https://books.google.by/books?id=BjNZXwH7HlkC&pg=PA2&redir_esc=y#v=onepage&q&f=false - Heath, Steve (2003). Embedded systems design. EDN series for design engineers (2 ed.). Newnes. p. 2. ISBN 978-0-7506-5546-0. "An embedded system is a microprocessor based system that is built to control a function or a range of functions."
% https://stackoverflow.com/questions/5281848/what-are-the-five-most-commonly-used-real-time-operating-systems - примеры RTOS-ов
% https://en.wikipedia.org/wiki/Bare_machine - bare metal
% https://www.embedded.com/electronics-blogs/barr-code/4027479/Real-men-program-in-C - диаграмма с языками для embedded
