\section{СИСТЕМНОЕ ПРОЕКТИРОВАНИЕ}
\label{sec:sys}

Изучив теоретические аспекты разрабатываемой системы и выработав
список требований необходимых для разработки системы, разбиваем систему
на функциональные блоки(модули). Это необходимо для обеспечения гибкой
архитектуры. Такой подход позволяет изменять или заменять модули без
изменения всей системы в целом.

В разрабатываемом веб-приложении можно выделить следующие
блоки:

\begin{itemize}
    \item блок загрузки данных;
    \item блок преобразования данных;
    \item блок передачи сообщений;
    \item блок приёма сообщений;
    \item блок анализа данных;
    \item блок сохранения данных;
    \item блок визуализации данных.
\end{itemize}
Взаимосвязь между основными компонентами проекта отражена на структурной схеме
ГУИР.400201.079 С1.

\subsection{Блок загрузки данных}

Блок загрузки данных предназначен для получения информацию из удалённого источника.
В качестве такого источника выступает сервис OpenWeatherMap~\cite{open_weather_map_index}.
Данный сервис поддерживает множество application programming interfaces (API), для выборки данных.
Конкретно для текущих целей используются API для следующих данных:
\begin{itemize}
    \item индекс моноооксида углерода ($ \text{CO} $);
    \item индекс озона ($ \text{O}_{\text{3}} $);
    \item индекс диоксида серы ($ \text{SO}_{\text{2}} $);
    \item индекс диоксида азота ($ \text{NO}_{\text{2}} $).
\end{itemize}

Рассматриваемый блок использует каждый вышеперечисленный API для получения данных.
Посредством использования HyperText Tranfer Protocol (HTTP) запроса, блок получает необходимые данные.
Данные предоставляются в формате JavaScript Object Notation (JSON).
Данный формат позволяет легко получить необходимые поля предоставляемых данных.

Из-за использования модульной структуры данный блок можно заменить блоком, который будет предоставлять локальные данные.
Таким образом при отсутствии должного количества данных, есть возможность воспользоваться локальным датасетом.
Полученные данным блоком данные отправляются в блок преобразования данных.

\subsection{Блок преобразования данных}

Блок преобразования данных осуществляет преобразования исходных данных, полученных в формате JSON в необходимые для дальнейшего анализа данные.
Текущая структура JSON данных, предоставляемая сервисом OpenWeatherMap имеет множество полей, которые несут дополнительную информацию.
Например, в таком формате хранятся данные о исследуемом индексе на различных высотах с разным атмосферным давлением.
Как указано в документации к данному сервису, для анализа рекомендуется брать значения между 215 и 0.00464 гектопаскалей (hPa).
Поэтому целью рассматриваемого блока и является выбор необходимых значений из предоставляемого формата.
Для каждого индекса используется свой трансформатор, так как структура JSON для некоторых индексов отличается.
Все эти данные преобразуются в цифровые значения и перенаправляются в блок передачи сообщений.

\subsection{Блок передачи сообщений}

Блок передачи данных предназначен для отправки подготовленных данных в очередь сообщений.
Данная операция необходима для того, чтобы абсолютно любые компоненты системы смогли получить обработанные данные и обработать их своим способом.
В качестве очереди сообщений выступает, описанная в предыдущем разделе, платформа kafka.
Данная платформа позволит остальным компонентам системы своевременно получить подготовленные данные и начать их обработку.

Одними из важнейших особенностей платформы kafka являются наличие топиков, и представление сообщения как пары ключ-значение.
Также на каждый предоставляемый индекс будет использоваться отдельны топик.
И из-за использования системы подписки, остальные компоненты системы могут отлавливать только интересующие их индексы.
Такой метод позволяет использовать один источник данных для всех компонентов системы.
Тем самым обеспечив разрабатываемой системе гибкость и вертикальную расширяемость.

Ещё одним не маловажным фактором выбором kafka является её отказоустойчивость.
Все эти сообщения будут храниться на кластере.
Тем самым, все отправленные данные не потеряются при внезапном отключении машины в кластере.


\subsection{Блок приёма сообщений}

Блок приёма сообщений отвечает за своевременное принятие сообщений из платформы kafka.
Как только данные были получены, они сразу же начинают обработку.
Такое поведение называется потоковой обработкой (англ. stream processing).




Обрабатывают входную относительно библиотеки информацию. Данный модуль
будет перехватывать все системные сообщения, которые будут сгенерированы
следующими компонентами TI-RTOS:
\begin{itemize}
    \item \texttt{xdc.runtime.Log};
    \item \texttt{xdc.runtime.System}.
\end{itemize}

Компонент \texttt{xdc.runtime.Log} предназначен для быстрой генерации сообщений.
Он поддерживает вывод сообщений, схожий с функцией \texttt{printf}, которая
определена в стандартной библиотеке ввода-вывода языка C,
но требует на этапе компиляции указать количество передаваемых аргументов.
Это значение ограничено сверху числом шесть в самой библиотеке.

Во время выполнения при вызове функции \texttt{Log\_warning3} она сохраняет
сообщение в буфер, не выполняя его преобразование в строку для экономии времени.
Затем, во время работы самой низкоприоритетной задачи, которая выполняется
во время простоя системы, происходит разбор сохраненного ранее сообщения,
генерирование текста сообщения с помощью функции \texttt{sprintf} или ее аналогов
и передается далее для обработки. По умолчанию какая-либо обработка отсутствует.

Данный способ вывода приведет к потере части сообщений при отсутствии в системе
свободного процессорного времени, но позволит практически не влиять на работу
системы в целом. Для сравнения, вывод системных сообщений с использованием
функции \texttt{printf} занимает до \num{10000} циклов на DSP процессорах,
тогда как использование функции \texttt{Log\_info6} -- всего
\num{40} циклов во время работы системы.
% MA TI-RTOS ref. guide, page 3-18 (10k vs 40 cycles)

\nomenclaturex{DSP}{Digital Signal Processing}{цифровая обработка сигналов}

Компонент \texttt{xdc.runtime.System} предназначен для обычной генерации сообщений.
В отличии от стандартной библиотеки ввода-вывода языка \texttt{C} позволяет
определить способ отправки информации к пользователю либо выбрать из уже
определенных в TI-RTOS. Сетевой стек, поставляемый с операционной системой,
использует именно этот модуль для отображения информации пользователю.

Таким образом, данный модуль будет производить сбор пользовательских сообщений
и сообщений операционной системы для каждого ядра процессора и передавать
собранную информацию далее с помощью модуля межъядерного взаимодействия.

\subsection{Модуль локального сбора статистики}

Модуль локального сбора статистики является еще одним модулем, который получает
входную относительно библиотеки информацию. Данный модуль будет основан на
следующих компонентах TI-RTOS:
\begin{itemize}
    \item \texttt{ti.sysbios.knl.Task};
    \item \texttt{ti.sysbios.utils.Load}.
\end{itemize}

Компонент \texttt{ti.sysbios.knl.Task} используется для статического и
динамического создания задач в TI-RTOS. Через его API можно получить информацию
обо всех существующих на момент запроса задачах в системе. С помощью этого
компонента системный планировщик задач выполняет свою основную функцию.

Компонент \texttt{ti.sysbios.utils.Load} является узкоспециализированным.
Он предназначен для сбора информации о загрузке процессора разными задачами,
а также видами прерываний. С его помощью локальный модуль сбора статистики
по каждому окончанию обновления информации будет производить ее сбор,
упаковывать ее в специальную структуру фиксированного размера и передавать
для дальнейшей обработки с помощью модуля межъядерного взаимодействия.
Также в этом компоненте устанавливается период сбора статистической информации
о загрузке процессора.

Таким образом, описанный выше модуль будет производить сбор информации
о загрузке ядра, на котором запущен, и передавать ее для дальнейшей обработки
с помощью модуля межъядерного взаимодействия.

% Первый будет использоваться для получения системных объектов всех задач в системе,
% а второй -- для получения информации об использовании процессорного времени
% HWI, SWI и каждой существующей в системе задачей. Период обновления информации
% об используемом процессорном времени и функция, которая будет вызвана
% по завершению расчетов, также задаются в \texttt{ti.sysbios.utils.Load}.

\subsection{Модуль межъядерного взаимодействия}

Модуль межъядерного взаимодействия является одним из важных модулей при работе
с мультиядерными системами. Именно он позволяет обеспечивать атомарный
обмен данными как между гомогенными ядрами системы (например, с DSP ядра
процессора на DSP ядро процессора), так и между гетерогенными (с DSP ядра
процессора на ядро процессора с архитектурой ARM). Данный модуль будет
взаимодействовать со следующими компонентами TI-RTOS:
\begin{itemize}
    \item \texttt{ti.sdo.ipc.MessageQ};
    \item \texttt{ti.sdo.ipc.heaps.HeapMemMP};
    \item \texttt{ti.sdo.ipc.SharedRegion}.
\end{itemize}

Компонент \texttt{ti.sdo.ipc.MessageQ} управляет межъядерной пересылкой сообщений.
Поскольку пересылаемые сообщения могут быть разного размера, требуется иметь
кучу, в которой будет производиться выделение буферов для отправляемых сообщений.
Таким образом, между ядрами будет требоваться только пересылка адреса и размера
выделенного буфера для каждого сообщения. Это и позволит пересылать сообщения
переменной длины с помощью сообщений постоянной длины.

Компонент \texttt{ti.sdo.ipc.heaps.HeapMemMP} будет использован в качестве
управляющего для межъядерной кучи. Он позволяет выделять сообщения переменной длины
для их дальнейшей отправки на другое ядро.

% crunch for texttt names using without hyphenation
\newdimen\origiwstr
\origiwstr=\fontdimen3\font
\fontdimen3\font=2\origiwstr

Компонент \texttt{ti.sdo.ipc.SharedRegion} потребуется для управления выделенным
диапазоном памяти (регионом), который должен быть доступен со всех ядер
микропроцессора. Компонент также зависит от \texttt{ti.sdo.ipc.GateMP},
\texttt{ti.sdo.ipc.Notify} и вышеупомянутого \texttt{ti.sdo.ipc.MessageQ},
которые используются системой для синхронизации процесса общения между
несколькими ядрами.

% crunch for texttt names using without hyphenation
\fontdimen3\font=\origiwstr

Также данный модуль будет обязан проводить идентификацию источника и
приемника сообщений. Первое может быть полезно на принимающей стороне, поскольку
с помощью этого можно производить ветвление логики (например, сообщения от
источника с этого же ядра будут проходить ускоренную обработку). Второе будет
полезно для работы системы обмена сообщения в целом: нужно определять, на какое
ядро отправлять посылку и какой конкретно из ожидающих межъядерного сообщения
задаче его отдавать на обработку.

Таким образом, данный модуль будет заниматься пересылкой сообщений между
несколькими ядрами и идентифицировать источники сообщений. Наличие этого
блока будет обязательными для систем с несколькими ядрами процессора.

\subsection{Модуль глобального сбора сообщений}

Модуль глобального сбора сообщений является своеобразным прокси-сервером,
который будет получать сообщения из одного модуля и по требованию пересылать
в один или несколько других модулей.

Для экономии операций копирования, которые могут быть переменной длины,
потребуется использовать специальный режим получения сообщений из модуля
межъядерного взаимодействия, который вернет буфер, указатель на который
использовался при передаче.

Для отправки информации требуетмым модулям будет использован компонент TI-RTOS
\texttt{ti.sysbios.knl.Mailbox}. Это программный буфер, операции
с которым являются атомарными, т.е. из одной задачи можно читать данные в буфер,
а из другой -- писать, не опасаясь некорректного переключения контекста.
Чтение из этого буфера является блокирующим, что позволяет в требуемой задачи
организовывать ожидание входящих сообщений без какого-либо дополнительного кода:
все будет решено через объекты синхронизации TI-RTOS внутри буфера. Аналогично
и с записью: если буфер будет заполнен, текущий поток выполнения может быть
заблокирован до появления там свободного места, чтобы произвести добавление.
Буфер работает по принципу очереди: первый пришел -- первый ушел.

Таким образом, модуль глобального сбора сообщений будет получать системные
сообщения со всех ядер процессора от модуля межъядерного взаимодействия,
а затем пересылать их в модуль взаимодействия с UART и модуль работы
с веб-сокетами.

\subsection{Модуль взаимодействия с UART}

Модуль взаимодействия с UART будет обеспечивать корректную настройку универсального
асинхронного приемопередатчика и атомарный доступ к нему. Поскольку один UART
будет зарезервирован под вывод отладочных сообщений, то его было бы правильно
отображать пользователю только в моменты, когда система простаивает. Для
этого понадобятся следующие компоненты TI-RTOS:
\begin{itemize}
    \item \texttt{ti.sysbios.knl.Idle};
    \item \texttt{ti.drv.UART}.
\end{itemize}

Компонент \texttt{ti.sysbios.knl.Idle} состоит из блока процедур, которые
выполняются во время простоя системы. По умолчанию в нем присутствует бесконечный
цикл, который и вызывает все остальные процедуры. Поэтому они должны быть
спроектированы без использования ожидания: если нужно подождать, то требуется
передать управление следующей низкоприоритетной процедуре, завершив выполнение
своей.

Компонент \texttt{ti.drv.UART} является аппаратно-зависимой частью системы.
Он поставляется вместе с TI-RTOS SDK для конкретного семейства микроконтроллеров
и микропроцессоров и предоставляет набор функций для настройки аппаратного
модуля. Но данный компонент не является потокобезопасным, поскольку содержит
минимальный набор компонентов для взаимодействия с аппаратурой через вызовы
функций, а не регистры.

Таким образом, данный модуль будет будет производить настройку UART и обеспечивать
атомарный доступ к нему для модуля глобального сбора сообщений и программиста в
виде API. Для передачи сообщений будет использоваться их буферизация
с последующим разбором очереди вывода во время простоя системы.

\subsection{Модуль глобального сбора статистики}

Модуль глобального сбора статистики выполняет функцию, аналогичную
модулю глобального сбора сообщений. В отличие от него, этот модуль
будет видоизменять принимаемые данные и преобразовывать их в формат JSON.
Это будет сделано для удобства обработки входных данных в модуле отображения
данных.

\nomenclaturex{JSON}{JavaScript Object Notation}{нотация объекта JavaScript}

Также этот модуль не предполагает отправку полученной статисти в UART.
Это объясняется тем, что данная информация имеет гораздо более презентативный
вид в представлении графиков. А данный способ представления информации
не поддерживается по умолчанию в терминальных программах.

Таким образом, модуль глобального сбора статистики будет получать специальным
образом сформированое сообщения с каждого ядра процессора от модуля межъядерного
взаимодействия, преобразовывать их в формат JSON и их в модуль работы
с веб-сокетами.

\subsection{Модуль HTTP-сервера}

Модуль HTTP-сервера будет взаимодействовать с веб-браузером пользователя.
Его основное назначение -- поддержка протокола HTTP, по которому он будет
отправлять пользователю модуль отображения данных, состоящий из файлов
следующих типов: html, css и js.

\nomenclaturex{HTML}{HyperText Markup Language}{язык гипертекстовой разметки}
\nomenclaturex{CSS}{Cascading Style Sheets}{каскадные таблицы стилей}
\nomenclaturex{JS}{JavaScript}{язык программирования JavaScript}

Первый тип файлов предназначен для отображения статической страницы с данными.
Это позволяет задать базовую разметку страницы, в которую потом могут вноситься
изменения.

Файлы css предназначены для придания странице более презентабельного вида,
позволяя настроить форматирование страницы.

Файлы js используются для добавления странице возможности динамического обновления
данных и взаимодействия со статически созданными на html странице элементами.

Для имплементации данного модуля будет использован компонент TI-RTOS
\texttt{ti.ndk.config.Http}. Это достаточно простой по своей внутренней
структуре HTTP-сервер, написанный на C. Из-за особенностей его внутренней
реализации может потребоваться его доработка для корректной работы сервера.
По умолчанию сервер может передавать только те файлы, что расположены
в специально файловой системе, сделанной в оперативной памяти.

Таким образом, основной функцией этого модуля является инициализация специальной
файловой системы для HTTP-сервера модулем отображения данных для передачи его
веб-браузеру конечного пользователя.

\subsection{Модуль сервера веб-сокетов}

Модуль сервера веб-сокетов предназначен для пересылки данных, сгенерированных
системой, модулю отображения данных во время его функционирования. Он является
важным и требует имплементации согласно стандарту, поскольку будет
взаимодействовать со сторонним программным обеспечением.

В стандарте~\cite{websock_rfc} указано, что веб-сокеты базируются на TCP-сокетах,
поэтому для имплементации данного модуля требуется использовать компонент TI-RTOS
\texttt{ti.ndk.config.Tcp}.

Для уменьшения нагрузки на прибор предполагается имплементация только части
стандарта: пересылаемые сообщения не будут маскироваться с помощью ключа,
всегда будут передаваться полностью, без разделения на подпакеты. Также
длина пересылаемых данных должна быть не более 65535 байт в одном пакете.

Сервер не будет поддерживать шифрование соединения (и, соответственно,
безопасную версию веб-сокетов) из-за требования к шифровнию каждого
отправляемого сообщения, что приведет к излишнему использованию процессорного
времени.

Таким образом, модуль сервера веб-сокетов будет получать данные от модулей
глобального сбора статистики и сообщений и передавать их модулю отображения
данных.

\subsection{Модуль отображения данных}

Модуль отображения данных является одним из модулей, с которым будет
взаимодействовать конечный пользователь.

Для уменьшения нагрузки на операционную систему реального времени было решено
создать одностраничное приложение. Таким образом будет уменьшено количество
запросов к HTTP-серверу, поскольку все возможные отображения уже будут загружены
в основном документе. Смена текущего отображения будет производиться с помощью
JavaScript-кода на веб-странице.

Пользователь будет получать модуль с помощью веб-браузера. Для этого потребуется
ввод IP-адреса прибора. Пользователю будут доступны три основные вкладки:
\begin{itemize}
    \item загруженность устройства;
    \item сообщения системы;
    \item задачи.
\end{itemize}

Во вкладке <<Загруженность устройства>> пользователь сможет увидеть общую загрузку
каждого ядра процессора. Там будет отображено, сколько в процентном соотношении
процессорного времени занимали аппаратные и программные прерывания, а также задачи.

Во вкладке <<Сообщения системы>> будут доступны все пришедшие из системы сообщения.
Они будут визуально различаться по уровню (системное, оповещение, предупреждение,
ошибка). Для каждого сообщения будет указано, на каком ядре оно было сгенерировано.

Для вкладки <<Задачи>> потребуется выбрать ядро процессора, для которого будет
производиться отображение информации в данный момент времени. В данной вкладке
пользователь сможет увидеть распределение процессорного времени по задачам системы,
а также статистику по каждой из них на небольшой период времени.

Для получения данных с целью их отображения на страницу будет использована
технология веб-сокетов. В данном модуле будет использован клиентский веб-сокет,
который имплементируется браузером. После установления соединения с модулем
сервера веб-сокетов, данный модуль будет получать сообщения, определять их тип
и отображать представленную там информацию для конечного пользователя.

Таким образом, модуль отображения данных является одним из самых важных с точки
зрения пользователя. Этот модуль будет полностью находится в прошивке прибора
и отправляться конечному пользователю по запросу модулем HTTP-сервера.
Динамическая информация с прибора будет поступать на веб-страницу через модуль
сервера веб-сокетов.

Все модули, изложенные в данном разделе, являются обязательными для
полного функционирования разрабатываемого модуля.
