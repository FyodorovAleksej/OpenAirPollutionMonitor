% For page mirroring
\newgeometry{left=3cm,right=1.5cm,top=2.0cm,bottom=2.695cm,twoside}

    \begin{center}
      Министерство образования Республики Беларусь\\[1em]
      Учреждение образования\\
      БЕЛОРУССКИЙ ГОСУДАРСТВЕННЫЙ УНИВЕРСИТЕТ \\
      ИНФОРМАТИКИ И РАДИОЭЛЕКТРОНИКИ\\[1em]
    \end{center}
      Факультет: ФКСИС. Кафедра: ЭВМ. \\
% Alt + 0171 => «
% Alt + 0187 => »
      Специальность: 40 02 01 «Вычислительные машины, системы и сети». \\
      Специализация: 40 02 01-01 «Проектирование и применение локальных
      компьютерных сетей».

      \begin{flushright}
        \begin{minipage}{0.4\textwidth}
          \MakeUppercase{Утверждаю}\\
          Заведующий кафедрой ЭВМ\\
          \underline{\hspace*{2.2cm}} Б.В.~Никульшин \\
          «\underline{\hspace*{1cm}}» \underline{\hspace*{2.5cm}} 2019 г.
        \end{minipage}\\[1em]
      \end{flushright}

    \begin{center}
      {ЗАДАНИЕ}\\
      {по дипломному проекту студента}\\
      {Фёдорова Алексея Сергеевича}\\[1em]
    \end{center}

    \begin{enumerate}[label=\textbf{\arabic*}, itemsep=1em,
      leftmargin=*,
      itemindent=\taskLiskSectionItemLeftIndent-\taskLiskSubsectionDelta]
      \item Тема проекта: «Программное средство для отслеживания и анализа показателей загрязнения окружающей среды на основе нейронной сети» -- утверждена приказом по университету от
      {5 апреля 2019 г. № 806-c}.
      \item Срок сдачи студентом законченного проекта: 1 июня 2019 г.
      \item Исходные данные к проекту:
      \begin{enumerate_nest_bold_num}
        \item Среда разработки: IntelliJ IDEA.
        \item Используемый брокер сообщений: Apache Kafka.
        \item Используемый сервер: Apache Zeppelin.
        \item Способ отображения информации: веб-страница.
      \end{enumerate_nest_bold_num}

      \item Содержание пояснительной записки (перечень подлежащих разработке вопросов):
      \begin{enumerate_for_empty}
        \item[] Введение. 1. Обзор литературы. 2. Системное проектирование.
        3. Функциональное проектирование. 4. Разработка программных модулей.
        5. Программа и методика испытаний. 6. Руководство пользователя.
        7. Экономическая часть. Заключение.
        Список использованных источников. Приложения.
      \end{enumerate_for_empty}
      \item Перечень графического материала (с точным указанием обязательных чертежей):
      \begin{enumerate_nest_bold_num}
        \item Вводный плакат. Плакат.
        \item Заключительный плакат. Плакат.
        \item Модуль сбора статистики в режиме реального времени. Схема структурная.
        \item Модуль сбора статистики в режиме реального времени. Диаграмма последовательности.
        \item Модуль сбора статистики в режиме реального времени. Схема программы.
        \item Модуль сбора статистики в режиме реального времени. Схема данных.
      \end{enumerate_nest_bold_num}
      \item Содержание задания по экономической части:
      <<Технико-экономическое обоснование разработки программного средства для отслеживания и анализа показателей загрязнения окружающей среды на основе нейронной сети>>
      \begin{enumerate_for_empty}
        \item[] \vspace{1em}
        % @{} for left margin removing in tabular
        \begin{tabular}{ @{}p{0.75\textwidth - \taskLiskSubsectionDelta}p{0.25\textwidth} }
         \MakeUppercase{Задание выдал} & А.А.~Горюшкин
        \end{tabular}
      \end{enumerate_for_empty}
    \end{enumerate}

    \begin{center}
      \vspace{1em}
      {КАЛЕНДАРНЫЙ ПЛАН}\\
      \vspace{1em}

      \begin{tabular}
                 {| >{\raggedright}m{0.43\textwidth}
                  | >{\centering}m{0.08\textwidth}
                  | >{\centering}m{0.18\textwidth}
                  | >{\centering\arraybackslash}m{0.19\textwidth}|}
        \hline
          \centering Наименование этапов дипломного проекта
        & Объем этапа, \% & Срок выполнения этапа & Примечания \\
        \hline
        Подбор и изучение литературы & 10 & 23.03 – 07.04 & \\ \hline
        Структурное проектирование & 10 & 07.04 – 19.04 & \\ \hline
        Функциональное проектирование & 20 & 19.04 – 28.04 & \\ \hline
        Разработка программных модулей & 30 & 28.04 – 08.05 & \\ \hline
        Программа и методика испытаний & 10 & 08.05 – 18.05 & \\ \hline
        Расчет экономической эффективности & 10 & 18.05 – 25.05 & \\ \hline
        Оформление пояснительной записки & 10 & 25.05 – 01.06 & \\ \hline
      \end{tabular}
    \end{center}

    \vspace{1em}
    \noindent
    Дата выдачи задания: 23 марта 2019 г.

    \vspace{1em}
    \noindent
    % @{} for left margin removing in tabular
    \begin{tabular}{ @{}p{0.75\textwidth}p{0.25\textwidth} }
      Руководитель & Е.А.~Сасин \\
    \end{tabular}

    \vspace{1em}
    \noindent
    \MakeUppercase{Задание принял к исполнению}\hspace*{1.5cm}\underline{\hspace*{2.5cm}}

  \newpage

\restoregeometry
