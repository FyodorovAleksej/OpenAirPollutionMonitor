\newcommand{\rub}{руб}

\section{ТЕХНИКО-ЭКОНОМИЧЕСКОЕ ОБОСНОВАНИЕ РАЗРАБОТКИ ПРОГРАММНОГО ОБЕСПЕЧЕНИЯ ДЛЯ АНАЛИЗА ЗАГРЯЗНЕНИЯ ОКРУЖАЮЩЕЙ СРЕДЫ НА ОСНОВЕ НЕЙРОННОЙ СЕТИ}

% Begin Calculations

%% config
\FPeval{\configRoundSigns}{2}
%% config end

%% conf 1.1 from manual
\FPeval{\valToaDelim}{80}
\FPeval{\valTbsDelim}{22}
\FPeval{\valTpDelim}{22}
\FPeval{\valTotlDelim}{4.5}
\FPeval{\valTdrDelim}{18}
\FPeval{\valTdoScale}{0.75}
\FPeval{\valW}{1.3}%%%%%%%%%%%%%%%%%%%%%%%%%
\FPeval{\valK}{0.8}%%%%%%%%%%%%%%%%%%%%%%%%%
\FPeval{\valC}{2.2}%%%%%%%%%%%%%%%%%%%%%%%%%
\FPeval{\valp}{0.35}%%%%%%%%%%%%%%%%%%%%%%%%%
%%%% operators in your code (approximately)
\FPeval{\valq}{1900}%%%%%%%%%%%%%%%%%%%%%%%%%

%% conf 1.4 from manual
\FPeval{\valNe}{0.085}%%%%%%%%%%%%%%%%%%%%%%%%%
\FPeval{\valkis}{0.9}
\FPeval{\valTSe}{0.25}
\FPeval{\valTSpk}{4200}%%%%%%%%%%%%%%%%%%%%%%%%%
\FPeval{\valky}{1}
\FPeval{\valkm}{1.05}
\FPeval{\valNpka}{10}
\FPeval{\valkro}{0.13}
\FPeval{\valNpla}{1.2}
\FPeval{\valkd}{3}
\FPeval{\valTSpl}{300}
\FPeval{\valkre}{0.05}
\FPeval{\valSpk}{1}%%%%%%%%%%%%%%%%%%%%%%%%%
\FPeval{\valkar}{13.6}
\FPeval{\valkkomf}{0.9}
\FPeval{\valkpov}{0.85}
\FPeval{\valtss}{8}
\FPeval{\valTsg}{305}

%% conf 1.2 from manual
\FPeval{\valKpr}{1.7}%%%%%%%%%%%%%%%%%%%%%%%%%
\FPeval{\valNd}{13}%%%%%%%%%%%%%%%%%%%%%%%%%
\FPeval{\valNsots}{34.6}
\FPeval{\valtmes}{170}
\FPeval{\valZPr}{490}%%%%%%%%%%%%%%%%%%%%%%%%%
\FPeval{\valkt}{3.54}
\FPeval{\valF}{1.18}%%%%%%%%%%%%%%%%%%%%%%%%%
%% crunch - was fixed by 1.4 from manual
% \FPeval{\valSmch}{11.3}

%% conf 1.3 from manual
\FPeval{\valNp}{0.27}%%%%%%%%%%%%%%%%%%%%%%%%%
\FPeval{\valNDS}{34.6}

%% conf 2.2 from manual
\FPeval{\valKz}{70}%%%%%%%%%%%%%%%%%%%%%%%%%
\FPeval{\valHz}{1}
\FPeval{\valdpz}{0.17}%%%%%%%%%%%%%%%%%%%%%%%%%
\FPeval{\valTr}{0}%%%%%%%%%%%%%%%%%%%%%%%%%
\FPeval{\valTviv}{0}%%%%%%%%%%%%%%%%%%%%%%%%%
\FPeval{\valk}{402875}%%%%%%%%%%%%%%%%%%%%%%%%%

%% conf 2.2 from manual (for analogue)
\FPeval{\valKza}{90}%%%---===
\FPeval{\valdpza}{0.18}%%%---===
\FPeval{\valTra}{0}%%%---===
\FPeval{\valTviva}{0.05}%%%---===

%% conf 2.3 from manual
\FPeval{\valSnp}{18}

%% conf 3 from manual
\FPeval{\valx}{1}%%%---===
\FPeval{\valE}{0.1}%%%---===

%% calc 1.1 from manual
\FPeval{\valQ}{clip(\valq * \valC * (1 + \valp))}

\FPeval{\valToa}{round((\valQ * \valW * \valK) / \valToaDelim, \configRoundSigns)}
\FPeval{\valTbs}{round(\valQ * \valK / \valTbsDelim, \configRoundSigns)}
\FPeval{\valTp}{round(\valQ * \valK / \valTpDelim, \configRoundSigns)}
\FPeval{\valTotl}{round(\valQ * \valK / \valTotlDelim, \configRoundSigns)}
\FPeval{\valTdr}{round(\valQ * \valK / \valTdrDelim, \configRoundSigns)}
\FPeval{\valTdo}{round(\valTdr * \valTdoScale, \configRoundSigns)}

\FPeval{\valTrz}{clip(\valToa + \valTbs + \valTp + \valTotl + \valTdr + \valTdo)}

%% calc 1.4 from manual
\FPeval{\valSe}{round(\valNe * \valkis * \valTSe, \configRoundSigns)}

\FPeval{\valTSpkb}{round(\valTSpk * \valky * \valkm, \configRoundSigns)}
\FPeval{\valApk}{round(\valTSpkb * \valNpka / 100, \configRoundSigns)}
\FPeval{\valRpk}{round(\valTSpkb * \valkro, \configRoundSigns)}

\FPeval{\valTSplb}{round(\valSpk * \valkd * \valTSpl, \configRoundSigns)}
\FPeval{\valApl}{round(\valTSplb * \valNpla / 100, \configRoundSigns)}
\FPeval{\valRpl}{round(\valTSplb * \valkre, \configRoundSigns)}

\FPeval{\valRar}{round(\valSpk * \valkd * \valkar * \valkkomf * \valkpov * 12, \configRoundSigns)}

\FPeval{\valFpk}{round(\valtss * \valTsg, \configRoundSigns)}

\FPeval{\valSmch}{round(\valSe + ((\valApk + \valRpk + \valApl + \valRpl + \valRar) / \valFpk), \configRoundSigns)}

%% calc 1.2 from manual
\FPeval{\valtchr}{round((\valZPr * \valkt)/(\valtmes), \configRoundSigns)}
\FPeval{\valZrz}{round(\valTrz * \valtchr * \valKpr * (1 + (\valNd / 100)) * (1 + (\valNsots / 100)), \configRoundSigns)}
\FPeval{\valZot}{round(\valTotl * \valSmch, \configRoundSigns)}
\FPeval{\valSpr}{round(\valZrz * \valF + \valZot, \configRoundSigns)}

%% calc 1.3 from manual
\FPeval{\valPr}{round(\valSpr * \valNp, \configRoundSigns)}
\FPeval{\valTSo}{clip(\valSpr + \valPr)}
\FPeval{\valTSpr}{round(\valTSo + (\valZrz + \valPr) * \valNDS / 100, \configRoundSigns)}

%% calc 2.2 from manual
\FPeval{\valTvv}{round((\valKz * \valHz) / 100, \configRoundSigns)}
\FPeval{\valTz}{round((\valTvv + \valTr + \valTviv) * (1 + \valdpz) / 60, \configRoundSigns)}

\FPeval{\valtchp}{\valtchr}
\FPeval{\valZp}{round((\valTz * \valk * \valtchp * \valKpr) * (1 + (\valNd / 100)) * (1 + (\valNsots / 100)), \configRoundSigns)}

\FPeval{\valZa}{round(\valTz * \valk * \valSmch, \configRoundSigns)}
\FPeval{\valZt}{clip(\valZp + \valZa)}

%% calc 2.2 from manual (for analogue)
\FPeval{\valTvva}{round((\valKza * \valHz) / 100, \configRoundSigns)}
\FPeval{\valTza}{round((\valTvva + \valTra + \valTviva) * (1 + \valdpza) / 60, \configRoundSigns)}

\FPeval{\valZpa}{round((\valTza * \valk * \valtchp * \valKpr) * (1 + (\valNd / 100)) * (1 + (\valNsots / 100)), \configRoundSigns)}

\FPeval{\valZaa}{round(\valTza * \valk * \valSmch, \configRoundSigns)}
\FPeval{\valZta}{clip(\valZpa + \valZaa)}

%% calc 2.3 from manual
\FPeval{\valPu}{round((\valZta - \valZt) * (1 - \valSnp / 100), \configRoundSigns)}

%% calc 3 from manual
% \FPeval{\valKz}{round(\valTSpkb * (1 - \valx * \valNpka / 100) * \valTz * \valk / \valFpk, \configRoundSigns)}
\FPeval{\valKo}{clip(\valKz + \valTSpr)}

\FPeval{\valEF}{round(\valPu - \valE * \valKo, \configRoundSigns)}
\FPeval{\valTv}{round(\valKo / \valPu, \configRoundSigns)}
\FPeval{\valTvRounded}{round(valTv, 0)}

% End Calculations

\subsection{Краткая характеристика программного продукта}

В результате дипломного проектирования был разработан проект, который позволяет осуществлять анализ данных о загрязнении окружающей среды из внешнего источника в режиме реального времени.
Из-за абстрагирования от источника данных, такой проект может использовать абсолютно любой источник данных.
Данная особенность предоставляет проекту высокую гибкость.
Также из-за возможности произведения вычислений на полноценном кластере появляется возможность обрабатывать огромные массивы данных, что позволит увеличить нагрузку на систему.
При использовании распределённой системы хранения обеспечивается надёжность хранения полученных данных.

Текущий источник данных в лице сервиса OpenWeatherMap не позволяет обеспечить высокую нагрузку и скорость данных.
Так как используемый API для этого сервиса на текущий момент ещё развивается, то он имеет некоторые ограничения.
Из-за этих ограничений, для полноценной работы сервиса пришлось эмулировать источник данных с помощью локальных данных.
Такой подход позволил оценить производительность системы при высокой частоте отправки сообщений и высокой нагрузке.

Для проверки работы на полноценном кластере был использован инструмент контейнеризации, который позволил на одной машине запустить несколько контейнеров, представляющих из себя компоненты кластера.
Также он позволил организовать связь между запущенными контейнерами, что позволило сэмулировать полноценный кластер, в котором все машины связаны между собой.

В результате полученный продукт обладает следующими качествами:
\begin{itemize}
  \item отказоустойчивость при использовании нескольких машин.
  \item высокая пропускная способность из-за использования параллельных вычислений.
  \item гибкий источник данных. Так как в его роли может выступать любой процесс, который будет отправлять данные в брокер сообщений.
  \item анализ в режиме реального времени из-за использования пакетной обработки данных. 
\end{itemize}

\subsection{Определение трудоемкости разработки программного продукта}

Оценка трудоемкости разработки может быть определена с помощью укрупненного метода.
Для этого требуется использовать следующую формулу:

\begin{equation}
  \label{eq:econ:Trz}
  \text{Т}_\text{рз} = \text{Т}_\text{оа} +
    \text{Т}_\text{бс} +
    \text{Т}_\text{п} +
    \text{Т}_\text{отл} +
    \text{Т}_\text{др} +
    \text{Т}_\text{до} ,
\end{equation}
\begin{explanationx}
  \item[где] $ \text{Т}_\text{оа} $ -- трудоемкость подготовки описания задачи и исследования алгоритма решения;
  \item $ \text{Т}_\text{бс} $ -- трудоемкость разработки блок-схемы алгоритма;
  \item $ \text{Т}_\text{п} $ -- трудоемкость программирования по готовой блок-схеме;
  \item $ \text{Т}_\text{отл} $ -- трудоемкость отладки программы на ПК;
  \item $ \text{Т}_\text{др} $ -- трудоемкость подготовки документации по задаче в рукописи;
  \item $ \text{Т}_\text{до} $ -- трудоемкость редактирования, печати и оформления документации по задаче.
\end{explanationx}

Трудоемкость описания задачи и исследования алгоритма решения может быть определена
по следующей формуле:

\begin{equation}
  \label{eq:econ:Toa}
  \text{Т}_\text{оа} = \frac{Q \cdot W \cdot K}{\valToaDelim},
\end{equation}
\begin{explanationx}
  \item[где] $ Q $ -- условное число операторов в разрабатываемом программном продукте;
  \item $ W $ -- коэффициент увеличения затрат труда вследствие недостаточного или
  некачественного описания задачи;
  \item $ K $ -- коэффициент квалификации разработчика алгоритмов и программ.
\end{explanationx}

Аналогично рассчитываются и оставшиеся неизвестные величины в формуле~(\ref{eq:econ:Trz}):

\begin{equation}
  \label{eq:econ:Tbs}
  \text{Т}_\text{бс} = \frac{Q \cdot K}{\num{\valTbsDelim}}
\end{equation}

\begin{equation}
  \label{eq:econ:Tp}
  \text{Т}_\text{п} = \frac{Q \cdot K}{\num{\valTpDelim}}
\end{equation}

\begin{equation}
  \label{eq:econ:Totl}
  \text{Т}_\text{отл} = \frac{Q \cdot K}{\num{\valTotlDelim}}
\end{equation}

\begin{equation}
  \label{eq:econ:Tdr}
  \text{Т}_\text{др} = \frac{Q \cdot K}{\num{\valTdrDelim}}
\end{equation}

\begin{equation}
  \label{eq:econ:Tdo}
  \text{Т}_\text{до} = \num{\valTdoScale} \cdot \text{Т}_\text{др}
\end{equation}

Условное число операторов $ Q $ можно рассчитать по формуле:

\removeEquantionBeforeSpace{}

\begin{equation}
  \label{eq:econ:Q}
  Q = q \cdot C \cdot ( 1 + p ),
\end{equation}
\begin{explanationx}
  \item[где] $ q $ -- число операторов в программе;
  \item $ C $ -- коэффициент сложности программы;
  \item $ p $ -- коэффициент коррекции программы в ходе ее разработки.
\end{explanationx}

Посчитав число операторов в программе, получим $ q = \num{\valq} $. Приняв $ C = \num\valC $,
а $ p = \num\valp $, рассчитаем условное число операторов $ Q $ по формуле~(\ref{eq:econ:Q}):

\begin{equation}
  \label{eq:econ:valQ}
  Q = \num\valq \cdot \num\valC \cdot ( 1 + \num\valp ) = \num\valQ
\end{equation}

Из-за стажа работы разработчика алгоритмов менее двух лет, получаем $ K = \num\valK $ и
$ W = \num\valW $. Эти величины понадобятся для расчета по формулам выше.

Подставив результат~(\ref{eq:econ:valQ}) в выражения~(\ref{eq:econ:Toa}) -- (\ref{eq:econ:Tdr}),
получим следующие результаты:

\begin{equation}
  \label{eq:econ:valToa}
  \text{Т}_\text{оа} = \frac{\num\valQ \cdot \num\valW \cdot \num\valK}
    {\valToaDelim} \approx \num\valToa
\end{equation}

\begin{equation}
  \label{eq:econ:valTbs}
  \text{Т}_\text{бс} = \frac{\num\valQ \cdot \num\valK}{\num{\valTbsDelim}}
    \approx \num\valTbs
\end{equation}

\begin{equation}
  \label{eq:econ:valTp}
  \text{Т}_\text{п} = \frac{\num\valQ \cdot \num\valK}{\num{\valTpDelim}}
    \approx \num\valTp
\end{equation}

\begin{equation}
  \label{eq:econ:valTotl}
  \text{Т}_\text{отл} = \frac{\num\valQ \cdot \num\valK}{\num{\valTotlDelim}}
    \approx \num\valTotl
\end{equation}

\begin{equation}
  \label{eq:econ:valTdr}
  \text{Т}_\text{др} = \frac{\num\valQ \cdot \num\valK}{\num{\valTdrDelim}}
    \approx \num\valTdr
\end{equation}

На основании расчетов выше и формулы~(\ref{eq:econ:Tdo}) получим значение
трудоемкости редактирования, печати и оформления документации по задаче:

\begin{equation}
  \label{eq:econ:valTdo}
  \text{Т}_\text{до} = \num\valTdoScale \cdot \num\valTdr \approx \num\valTdo
\end{equation}

Тогда на основании формулы~(\ref{eq:econ:Trz}) и расчетов ранее,
трудоемкость разработки составит:

\begin{equation}
  \label{eq:econ:valTrz}
  \text{Т}_\text{рз} =
    \num\valToa + \num\valTbs + \num\valTp + \num\valTotl + \num\valTdr +
    \num\valTdo = \num\valTrz
\end{equation}

\subsection{Определение стоимости машино-часа работы ПК}

Произведем расчет стоимости машино-часа работы ПК. Она определяется по формуле:

\begin{equation}
  \label{eq:econ:Smch}
  S_\text{мч} = \text{С}_\text{э} + \frac{\text{А}_\text{пк} + \text{Р}_\text{пк} +
  \text{А}_\text{пл} + \text{Р}_\text{пл} + \text{Р}_\text{ар}}{\text{Ф}_\text{пк}},
\end{equation}
\begin{explanationx}
  \item[где] $ \text{С}_\text{э} $ -- расходы на электроэнергию за час работы ПК, \rub;
  \item $ \text{А}_\text{пк} $ -- годовая величина амортизационных отчислений на реновацию ПК, \rub;
  \item $ \text{Р}_\text{пк} $ -- годовые затраты на ремонт и техническое обслуживание ПК, \rub;
  \item $ \text{А}_\text{пл} $ -- годовая величина амортизационных отчислений на реновацию производственных площадей, занимаемых ПК, \rub;
  \item $ \text{Р}_\text{пл} $ -- годовые затраты на ремонт и содержание производственных площадей, \rub;
  \item $ \text{Р}_\text{ар} $ -- годовая величина арендных платежей за помещение, занимаемое ПК, \rub;
  \item $ \text{Ф}_\text{пк} $ -- годовой фонд времени работы ПК, ч.
\end{explanationx}

Расходы на электроэнергию за час работы ПК определяются по формуле:

\begin{equation}
  \label{eq:econ:Se}
  \text{С}_\text{э} = N_\text{э} \cdot k_\text{ис} \cdot \text{Ц}_\text{э},
\end{equation}
\begin{explanationx}
  \item[где] $ N_\text{э} $ -- установленная мощность блока питания ПК, кВт;
  \item $ k_\text{ис} $ -- коэффициент использования энергоустановок по мощности;
  \item $ \text{Ц}_\text{э} $ -- стоимость киловатт-часа электроэнергии, \rub.
\end{explanationx}

Примем установленную мощность блока питания ПК равной \num\valNe кВт и коэффициент использования
энергоустановок по мощности \num\valkis. Стоимость киловатт-часа электроэнергии на момент расчетов
составила \num\valTSe { }\rub. Тогда расходы на электроэнергию за час работы ПК составляют:

\begin{equation}
  \label{eq:econ:valSe}
  \text{С}_\text{э} = \num\valNe \cdot \num\valkis \cdot \num\valTSe \approx \num\valSe \text{ \rub}
\end{equation}


Годовая величина амортизационных отчислений на реновацию ПК определяется по формуле:

\begin{equation}
  \label{eq:econ:Apk}
  \text{А}_\text{пк} = \text{Ц}_\text{пк}^\text{б} \cdot \frac{\text{Н}_\text{пк}^\text{а}}{100},
\end{equation}
\begin{explanationx}
  \item[где] $ \text{Ц}_\text{пк}^\text{б} $ -- балансовая стоимость ПК, \rub;
  \item $ \text{Н}_\text{пк}^\text{а} $ -- норма амортизационных отчислений на ПК, \%.
\end{explanationx}

Балансовая стоимость ПК определяется по формуле:

\begin{equation}
  \label{eq:econ:TSpk}
  \text{Ц}_\text{пк}^\text{б} = \text{Ц}_\text{пк} \cdot k_\text{у} \cdot k_\text{м},
\end{equation}
\begin{explanationx}
  \item[где] $ \text{Ц}_\text{пк} $ -- цена ПК на момент его выпуска, \rub;
  \item $ k_\text{у} $ -- коэффициент удорожания ПК;
  \item $ k_\text{м} $ -- коэффициент, учитывающий затраты на монтаж и транспортировку ПК.
\end{explanationx}

Приняв стоимость нового ПК равной \num\valTSpk { \rub}, коэффициент удорожания ПК равным \num\valky,
а коэффициент $ k_\text{м} $ равным \num\valkm, получим балансовую стоимость ПК:

\begin{equation}
  \label{eq:econ:valTSpk}
  \text{Ц}_\text{пк}^\text{б} = \num\valTSpk \cdot \num\valky \cdot \num\valkm \approx \num\valTSpkb
  \text{ \rub}
\end{equation}

Если взять норму амортизационных отчислений на ПК, равную \num\valNpka $ \% $, то годовая величина
амортизационных отчислений на реновацию ПК составит:

\begin{equation}
  \label{eq:econ:valApk}
  \text{А}_\text{пк} = \num\valTSpkb \cdot \frac{\num\valNpka}{100} = \num\valApk \text{ \rub}
\end{equation}

Годовые затраты на ремонт и техническое обслуживание ПК укрупненно могут быть определены по формуле:

\begin{equation}
  \label{eq:econ:Rpk}
  \text{Р}_\text{пк} = \text{Ц}_\text{пк}^\text{б} \cdot k_\text{ро},
\end{equation}
\begin{explanationx}
  \item[где] $ k_\text{ро} $ -- коэффициент, учитывающий затраты на ремонт и техническое обслуживание ПК.
\end{explanationx}

Подставив $ k_\text{ро} = \num\valkro $, найдем затраты на ремонт и техническое обслуживание ПК:

\begin{equation}
  \label{eq:econ:valRpk}
  \text{Р}_\text{пк} = \num\valTSpkb \cdot \num\valkro \approx \num\valRpk \text{ \rub}
\end{equation}

Годовая величина амортизационных отчислений на реновацию производственных площадей, занятых ПК,
определяется по формуле:

\begin{equation}
  \label{eq:econ:Apl}
  \text{А}_\text{пл} = \text{Ц}_\text{пл}^\text{б} \cdot \frac{\text{Н}_\text{пл}^\text{а}}{100},
\end{equation}
\begin{explanationx}
  \item[где] $ \text{Ц}_\text{пл}^\text{б}  $ -- балансовая стоимость площадей, \rub;
  \item $ \text{Н}_\text{пл}^\text{а}  $ -- норма амортизационных отчислений на
  производственные площади, \%.
\end{explanationx}

Балансовая стоимость площадей определяется по формуле:

\begin{equation}
  \label{eq:econ:TSplb}
  \text{Ц}_\text{пл}^\text{б} = S_\text{пк} \cdot k_\text{д} \cdot \text{Ц}_\text{пл},
\end{equation}
\begin{explanationx}
  \item[где] $ S_\text{пк} $ -- площадь, занимаемая ПК, м\textsuperscript{2};
  \item $ k_\text{д} $ -- коэффициент, учитывающий дополнительную площадь;
  \item $ \text{Ц}_\text{пл} $ -- цена квадратного метра производственной площади, \rub.
\end{explanationx}

Если принять, что один ПК занимает \num\valSpk { м}\textsuperscript{2},
$ k_\text{д} = \num\valkd $, а цена одного квадратного метра
производственной площади равна \num\valTSpl { \rub}, тогда
балансовая стоимость площадей составит:

\begin{equation}
  \label{eq:econ:valTSplb}
  \text{Ц}_\text{пл}^\text{б} = \num\valSpk \cdot \num\valkd \cdot \num\valTSpl \approx
  \num\valTSplb \text{ \rub}
\end{equation}

Подставив $ \text{Н}_\text{пл}^\text{а} = \num\valNpla \% $, получим годовую величину
амортизационных отчислений на реновацию производственных площадей, занятых ПК:

\begin{equation}
  \label{eq:econ:valApl}
  \text{А}_\text{пл} = \num\valTSplb \cdot \frac{\num\valNpla}{100} \approx \num\valApl \text{ \rub}
\end{equation}

Годовые затраты на ремонт и содержание производственных площадей укрупненно
могут быть определены по формуле:

\begin{equation}
  \label{eq:econ:Rpl}
  \text{Р}_\text{пл} = \text{Ц}_\text{пл}^\text{б} \cdot k_\text{рэ},
\end{equation}
\begin{explanationx}
  \item[где] $ k_\text{рэ} $ -- коэффициент, учитывающий затраты на ремонт и эксплуатацию
  производственных площадей.
\end{explanationx}

Подставив $ k_\text{рэ} = \num\valkre $, получим:

\begin{equation}
  \label{eq:econ:valRpl}
  \text{Р}_\text{пл} = \num\valTSplb \cdot \num\valkre \approx \num\valRpl \text{ \rub}
\end{equation}

Годовая величина арендных платежей за помещение, занимаемое ПК,
рассчитывается по формуле:

\begin{equation}
  \label{eq:econ:Rar}
  \text{Р}_\text{ар} = S_\text{пк} \cdot k_\text{д} \cdot k_\text{ар} \cdot k_\text{комф}
  \cdot k_\text{пов} \cdot 12,
\end{equation}
\begin{explanationx}
  \item[где] $ k_\text{ар} $ -- ставка арендных платежей за помещение;
  \item $ k_\text{комф} $ -- коэффициент комфортности помещения;
  \item $ k_\text{пов} $ -- повышающий коэффициент, учитывающий географическое размещение площади.
\end{explanationx}

Предположим, что ставка арендных платежей за помещение составляют \num\valkar
{ \rub}/м\textsuperscript{2}. Коэффициент комфортности возьмем равным \num\valkkomf, поскольку
он соответствует помещениям, расположенным в цокольном этаже. Из-за расположения арендуемых
зданий в городе Минске, $ k_\text{пов} = \num\valkpov $. Тогда годовая величина арендных
платежей за помещение, занимаемое ПК, составит:

\begin{equation}
  \label{eq:econ:valRar}
  \text{Р}_\text{ар} = \num\valSpk \cdot \num\valkd \cdot \num\valkar \cdot \num\valkkomf
  \cdot \num\valkpov \cdot 12 \approx \num\valRar \text{ \rub}
\end{equation}

Годовой фонд времени работы ПК определяется исходя из режима ее работы
и может быть рассчитан по формуле:

\begin{equation}
  \label{eq:econ:Fpk}
  \text{Ф}_\text{пк} = t_\text{сс} \cdot T_\text{сг},
\end{equation}
\begin{explanationx}
  \item[где] $ t_\text{сс} $ -- среднесуточная фактическая загрузка ПК, ч;
  \item $ T_\text{сг} $ -- среднее количество дней работы ПК в год.
\end{explanationx}

Приняв, что среднесуточная фактическая нагрузка составляет
\insertNumSomethingText{\valtss}{час}{часа}{часов},
а в году в среднем
\insertNumSomethingText{\valTsg}{рабочий день}{рабочих дня}{рабочих дней},
рассчитаем годовой фонд времени работы ПК:

\begin{equation}
  \label{eq:econ:valFpk}
  \text{Ф}_\text{пк} = \num\valtss \cdot \num\valTsg = \num\valFpk
\end{equation}

Тогда стоимость машино-часа работы ПК составит:

\begin{equation}
  \label{eq:econ:Smch}
  S_\text{мч} = \num\valSe + \frac{\num\valApk + \num\valRpk + \num\valApl +
  \num\valRpl + \num\valRar}{\num\valFpk} \approx \num\valSmch \text{ \rub}
\end{equation}

\subsection{Определение себестоимости создания программного продукта}

Для определения себестоимости создания программного продукта необходимо определить затраты
на заработную плату разработчика по формуле:

\begin{equation}
  \label{eq:econ:Zrz}
  \text{З}_{\text{рз}} =
  \text{Т}_{\text{рз}} \cdot
  t_{\text{чр}} \cdot
  \text{К}_{\text{пр}} \cdot
  \left(\text{1} + \frac{\text{Н}_{\text{д}}}{\text{100}}\right) \cdot
  \left(\text{1} + \frac{\text{Н}_{\text{соц}}}{\text{100}}\right),
\end{equation}
\begin{explanationx}
  \item[где] $ \text{Т}_\text{рз} $ -- трудоемкость разработки программного продукта, чел-ч;
  \item $ t_\text{чр} $ -- среднечасовая ставка работника, осуществлявшего разработку программного продукта, \rub;
  \item $ \text{К}_\text{пр} $ -- коэффициент, учитывающий процент премий
    в организации-раз\-ра\-бот\-чи\-ке;
  \item $ \text{Н}_\text{д} $ -- норматив дополнительной заработной платы;
  \item $ \text{Н}_\text{соц} $ -- норматив отчислений от фонда оплаты труда.
\end{explanationx}

Среднечасовая ставка работника определяется исходя из единой тарифной системы оплаты труда
в Республике Беларусь по следующей формуле:

\begin{equation}
  \label{eq:econ:tchr}
  t_\text{чр} = \frac{\text{ЗП}_\text{1р} \cdot k_\text{т}}{t_\text{мес}},
\end{equation}
\begin{explanationx}
  \item[где] $ \text{ЗП}_\text{1р} $ -- среднемесячная заработная плата работника первого разряда;
  \item $ k_\text{т} $ -- тарифный коэффициент работника соответствующего разряда;
  \item $ t_\text{мес} $ -- среднее нормативное количество рабочих часов в месяце.
\end{explanationx}

Среднемесячную заработную плату работника первого разряда возьмем в размере $ \num\valZPr $ \rub.
Из-за использования работников первой категории 11 разряда, тарифный коэффициент работника
$ k_\text{т} $ составит $ \num\valkt $. Учитывая, что в среднем в месяце $ \num\valtmes $
рабочих часов, на основе формулы~(\ref{eq:econ:tchr}) рассчитаем среднечасовую ставку работника,
который осуществляет разработку программного продукта:

\clearpage
\removeEquantionBeforeSpace[1.5]

\begin{equation}
  \label{eq:econ:valtchr}
  t_\text{чр} = \frac{\num\valZPr \cdot \num\valkt}{\num\valtmes} \approx \num\valtchr \text{ \rub}
\end{equation}

На сегодняшний день норматив отчислений от фонда оплаты труда составляет $ \num\valNsots \% $.
Предположим, что $\text{К}_\text{пр} = \num\valKpr $ и $\text{Н}_\text{д} = \num\valNd \%$.
На основании расчетов выше и формулы~(\ref{eq:econ:Zrz}) найдем затраты на заработную плату
разработчика:

\begin{equation}
  \label{eq:econ:valZrz}
  \text{З}_\text{рз} = \num\valTrz \cdot \num\valtchr \cdot \num\valKpr \cdot
    \left( 1 + \frac{\num\valNd}{\num{100}} \right) \cdot
    \left( 1 + \frac{\num\valNsots}{\num{100}} \right) \approx \num\valZrz \text{ \rub}
\end{equation}

Также в себестоимость разработки программного продукта включаются также затраты
на его отладку в процессе создания. Для определения их величины необходимо
рассчитать стоимость машино-часа работы ПК,
на котором осуществлялась отладка.
Данная величина соответствует величине арендной платы за час работы
ПК и будет определена ниже.
Затраты на отладку программы определяются по формуле:

\begin{equation}
  \label{eq:econ:Zot}
  \text{З}_\text{от} = \text{Т}_\text{отл} \cdot S_\text{мч},
\end{equation}
\begin{explanationx}
  \item[где] $ \text{Т}_\text{отл} $ -- трудоемкость отладки программы, ч;
  \item $ S_\text{мч} $ -- стоимость машино-часа работы ПК, \rub/ч.
\end{explanationx}

Подставив все известные данные в формулу~(\ref{eq:econ:Zot}), получим:

\begin{equation}
  \label{eq:econ:valZot}
  \text{З}_\text{от} = \num\valTotl \cdot \num\valSmch \approx \num\valZot \text{ \rub},
\end{equation}

Себестоимость разработки программного продукта определяется по формуле:

\begin{equation}
  \label{eq:econ:Spr}
  \text{С}_\text{пр} = \text{З}_\text{рз} \cdot F + \text{З}_\text{от},
\end{equation}
\begin{explanationx}
  \item[где] $ F $ -- коэффициент накладных расходов проектной организации без учета эксплуатации ПК.
\end{explanationx}

Предположив, что $ F = \num\valF $, рассчитаем себестоимость разработки
по формуле~(\ref{eq:econ:Spr}):

\begin{equation}
  \label{eq:econ:valSpr}
  \text{С}_\text{пр} = \num\valZrz \cdot \num\valF + \num\valZot \approx \num\valSpr \text{ \rub}
\end{equation}

\subsection{Определение оптовой и отпускной цены программного продукта}

Оптовая цена складывается из себестоимости создания ПП и плановой прибыли на программу.
Оптовая цена определяется по формуле:

\begin{equation}
  \label{eq:econ:TSo}
  \text{Ц}_\text{о} = \text{С}_\text{пр} + \text{П}_\text{р},
\end{equation}
\begin{explanationx}
  \item[где] $ \text{П}_\text{р} $ -- плановая прибыль на программу, \rub.
\end{explanationx}

Плановая прибыль на программу определяется по формуле:

\begin{equation}
  \label{eq:econ:Pr}
  \text{П}_\text{р} = \text{С}_\text{пр} \cdot \text{Н}_\text{п},
\end{equation}
\begin{explanationx}
  \item[где] $ \text{Н}_\text{п} $ -- норма прибыли проектной организации.
\end{explanationx}

Предположим, что норма прибыли организации составляет $ \num\valNp $. Тогда плановая прибыль
на программу составит:

\begin{equation}
  \label{eq:econ:valPr}
  \text{П}_\text{р} = \num\valSpr \cdot \num\valNp \approx \num\valPr \text{ \rub}
\end{equation}

Оптовая цена будет составлять:

\begin{equation}
  \label{eq:econ:valTSo}
  \text{Ц}_\text{о} = \num\valSpr + \num\valPr = \num\valTSo \text{ \rub}
\end{equation}

Отпускная цена программы определяется по формуле:

\begin{equation}
  \label{eq:econ:TSpr}
  \text{Ц}_\text{пр} = \text{Ц}_\text{о} + (\text{З}_\text{рз} + \text{П}_\text{р}) \cdot \text{НДС},
\end{equation}
\begin{explanationx}
  \item[где] $ \text{НДС} $ -- ставка налога на добавленную стоимость.
\end{explanationx}

На сегодняшний день в Республике Беларусь $ \text{НДС} = \num\valNDS\% $, поэтому отпускная цена
программы составит:

\begin{equation}
  \label{eq:econ:valTSpr}
  \text{Ц}_\text{пр} = \num\valTSo + (\num\valZrz + \num\valPr) \cdot \num\valNDS \% \approx
    \num\valTSpr \text{ \rub}
\end{equation}

\subsection{Определение годовых текущих затрат, связанных с эксплуатацией задачи}

Для расчета годовых текущих затрат, связанных с эксплуатацией ПП,
необходимо определить время решения данной задачи на ПК.
Время решения задачи на ПК определяется по формуле:

\begin{equation}
  \label{eq:econ:Tz}
  \text{Т}_\text{з} = (T_\text{вв} + T_\text{р} + T_\text{выв}) \cdot \frac{1 + d_\text{пз}}{60},
\end{equation}
\begin{explanationx}
  \item[где] $ T_\text{вв} $ -- время ввода в ПК исходных данных, необходимых для решения задачи, мин;
  \item $ T_\text{р} $ -- время вычислений, мин;
  \item $ T_\text{выв} $ -- время вывода результатов решения задачи, мин;
  \item $ d_\text{пз} $ -- коэффициент, учитывающий подготовительно-заключительное время.
\end{explanationx}

Время ввода в ПК исходных данных может быть определено по формуле:

\clearpage
\removeEquantionBeforeSpace[1.5]

\begin{equation}
  \label{eq:econ:Tvv}
  T_\text{вв} = \frac{K_z \cdot H_z}{100},
\end{equation}
\begin{explanationx}
  \item[где] $ K_z $ -- среднее количество знаков, набираемых с клавиатуры при вводе исходных данных;
  \item $ H_z $ -- норматив набора 100 знаков, мин.
\end{explanationx}

Ввод исходных данных в ПК подразумевает ввод парольного слова, которое будет в дальнейшем анализироваться. Для определенности возьмем среднее количество знаков, набираемых с клавиатуры, в размере \num\valKz { }символов.
Предположив, что норматив набора 100 знаков составляет \num\valHz { }минут, получим $ T_\text{вв} $:

\begin{equation}
  \label{eq:econ:valTvv}
  T_\text{вв} = \frac{\num\valKz \cdot \num\valHz}{100} \approx \num\valTvv
\end{equation}

Времена вычислений и вывода результатов составляют доли секунд, поэтому в сравнении с $ T_\text{вв} $
равны нулю. Взяв величину $ d_\text{пз} = \num\valdpz $, рассчитаем время решения задачи:

\begin{equation}
  \label{eq:econ:valTz}
  \text{Т}_\text{з} = (\num\valTvv + \num\valTr + \num\valTviv) \cdot \frac{1 + \num\valdpz}{60}
  \approx \num\valTz \text{ ч}
\end{equation}

На основе рассчитанного времени решения задачи может быть определена
заработная плата пользователя данного программного продукта. Затраты на заработную плату
пользователя программного продукта определяются по формуле:

\begin{equation}
  \label{eq:econ:Zp}
  \text{З}_{\text{п}} =
  \text{Т}_{\text{з}} \cdot
  k \cdot
  t_{\text{чп}} \cdot
  \text{К}_{\text{пр}} \cdot
  \left(\text{1} + \frac{\text{Н}_{\text{д}}}{\text{100}}\right) \cdot
  \left(\text{1} + \frac{\text{Н}_{\text{соц}}}{\text{100}}\right),
\end{equation}
\begin{explanationx}
  \item[где] $ \text{Т}_\text{з} $ -- время решения задачи на ПК, ч;
  \item $ k $  -- периодичность решения задачи в течение года, раз/год;
  \item $ t_\text{чп} $ -- среднечасовая ставка пользователя программы, \rub.
\end{explanationx}

Пользователями разрабатываемого продукта также будут являться
разработчиками аналогичного уровня, поэтому $ t_\text{чп} = t_\text{чр} = \num\valtchp \text{ \rub} $.
Число использования фреймворка зависит от пользовательской базы, а также от особенностей приложения, в котором используется. Если принять, что ежедневно приложением пользуются от 700 до 1400 пользователей, а фреймворк используется каждый раз при использовании приложения, 
то $ k = \valk $. Тогда затраты на заработную плату пользователя программного продукта составят:

\begin{equation}
  \label{eq:econ:valZp}
  \text{З}_{\text{п}} = \num\valTz \cdot \num\valk \cdot \num\valtchp \cdot \num\valKpr \cdot
  \left(\text{1} + \frac{\num\valNd}{\text{100}}\right) \cdot
  \left(\text{1} + \frac{\num\valNsots}{\text{100}}\right) \approx \num\valZp \text{ \rub}
\end{equation}

В состав затрат, связанных с решением задачи включаются также затраты, связанные с эксплуатацией ПК.
Затраты на оплату аренды ПК для решения задачи определяются по следующей формуле:

% \clearpage
% \removeEquantionBeforeSpace[1.5]

\begin{equation}
  \label{eq:econ:Za}
  \text{З}_{\text{а}} = \text{Т}_{\text{з}} \cdot k \cdot S_\text{мч}
\end{equation}

Стоимость одного машино-часа для пользователей и для разработчиков также будет совпадать
по вышеупомянутой причине. Поэтому $ \text{З}_\text{а} $ составит:

\begin{equation}
  \label{eq:econ:valZa}
  \text{З}_{\text{а}} = \num\valTz \cdot \num\valk \cdot \num\valSmch \approx \num\valZa \text{ \rub}
\end{equation}

Годовые текущие затраты, связанные с эксплуатацией задачи, определяются по формуле:

\begin{equation}
  \label{eq:econ:Zt}
  \text{З}_{\text{т}} = \text{З}_\text{п} + \text{З}_\text{а} =
  \num\valZp + \num\valZa = \num\valZt \text{ \rub}
\end{equation}

\subsection{Определение годовых затрат при решении задачи с помощью аналога}

Для расчета годовых затрат, связанных с эксплуатацией аналога,
необходимо определить время решения данной задачи на ПК.
Время решения задачи на ПК $ \text{Т}_\text{за} $ определяется по формуле,
аналогичной (\ref{eq:econ:Tz}):

\begin{equation}
  \label{eq:econ:Tza}
  \text{Т}_\text{за} = (T_\text{вва} + T_\text{ра} + T_\text{выва}) \cdot \frac{1 + d_\text{пза}}{60}
\end{equation}

Ввод исходных данных в ПК в этом случае также подразумевает ввод парольного слова. Для определенности, возьмем эту
величину $K_{za}$ в размере \num\valKza { }символов.
Предположив, что норматив набора 100 знаков также составляет \num\valHz { }минут,
в соответствии с формулой (\ref{eq:econ:Tvv}) получим время ввода исходных данных
для аналога $ T_\text{вва} $ :

\begin{equation}
  \label{eq:econ:valTvva}
  T_\text{вва} = \frac{K_{za} \cdot H_z}{100} = \frac{\num\valKza \cdot \num\valHz}
  {100} \approx \num\valTvva
\end{equation}

Время вычислений при использовании аналога $T_\text{ра}$ будет пренебрежительно малым. Время вывода результатов $T_\text{выва}$ зависит от скорости интернет соединения, поэтому будет колебаться в пределах от 1 секунды до 6 секунд, то есть составит приблизительно \valTviva { }минут.

Коэффициент $ d_\text{пза} $, учитывающий подготовительно-заключительное время,
составит \valdpza. Рассчитаем время решения задачи для аналога:

\begin{equation}
  \label{eq:econ:valTza}
  \text{Т}_\text{за} = (\num\valTvva + \num\valTra + \num\valTviva) \cdot \frac{1 + \num\valdpza}{60}
  \approx \num\valTza \text{ ч}
\end{equation}

На основе рассчитанного времени решения задачи может быть определена
заработная плата пользователя аналога программного продукта по формуле,
аналогичной (\ref{eq:econ:Zp}). Она будет отличаться от оригинальной только
временем решения задачи, найденного в формуле (\ref{eq:econ:valTza}):

\begin{equation}
  \label{eq:econ:Zpa}
  \text{З}_{\text{па}} =
  \text{Т}_{\text{за}} \cdot
  k \cdot
  t_{\text{чп}} \cdot
  \text{К}_{\text{пр}} \cdot
  \left(\text{1} + \frac{\text{Н}_{\text{д}}}{\text{100}}\right) \cdot
  \left(\text{1} + \frac{\text{Н}_{\text{соц}}}{\text{100}}\right)
\end{equation}


Затраты на заработную плату пользователя аналога программного продукта составят:

\begin{equation}
  \label{eq:econ:valZpa}
  \text{З}_{\text{па}} = \num\valTza \cdot \num\valk \cdot \num\valtchp \cdot \num\valKpr \cdot
  \left(\text{1} + \frac{\num\valNd}{\text{100}}\right) \cdot
  \left(\text{1} + \frac{\num\valNsots}{\text{100}}\right) \approx \num\valZpa \text{ \rub}
\end{equation}

В состав затрат, связанных с решением задачи включаются также затраты, связанные с эксплуатацией ПК.
Затраты на оплату аренды ПК для решения задачи определяются формуле, аналогичной
(\ref{eq:econ:Za}). Тогда $ \text{З}_\text{аа} $ составит:

\begin{equation}
  \label{eq:econ:valZaa}
  \text{З}_{\text{аа}} = \num\valTza \cdot \num\valk \cdot \num\valSmch \approx \num\valZaa \text{ \rub}
\end{equation}

Годовые текущие затраты, связанные с эксплуатацией аналога, определяются по формуле,
аналогичной (\ref{eq:econ:Zt}):

\begin{equation}
  \label{eq:econ:Zta}
  \text{З}_{\text{та}} = \text{З}_\text{па} + \text{З}_\text{аа}
  = \num\valZpa + \num\valZaa = \num\valZta \text{ \rub}
\end{equation}

% Тогда годовые текущие затраты составят:

% \begin{equation}
%   \label{eq:econ:Zta}
%   \text{З}_{\text{та}}
% \end{equation}

\subsection{Определение ожидаемого прироста прибыли в результате внедрения
программного продукта}

Ожидаемый прирост прибыли в результате внедрения задачи вместо ее
аналога укрупненно может быть определен по формуле:

\begin{equation}
  \label{eq:econ:Pu}
  \text{П}_{\text{у}} = (\text{З}_{\text{та}} - \text{З}_{\text{т}}) \cdot
  \left(1 - \frac{\text{С}_\text{нп}}{100}\right),
\end{equation}
\begin{explanationx}
  \item[где] $ \text{С}_\text{нп} $ -- ставка налога на прибыль.
\end{explanationx}

На момент расчетов ставка налога на прибыль составляла \num\valSnp \%. Прирост прибыли
составит:

\removeEquantionBeforeSpace[1.5]

\begin{equation}
  \label{eq:econ:valPu}
  \text{П}_{\text{у}} = (\num\valZta - \num\valZt) \cdot \left(1 - \frac{\num\valSnp}{100}\right)
  \approx \num\valPu
\end{equation}

\subsection{Расчет показателей эффективности использования программного продукта}

Для определения годового экономического эффекта от разработанной
программы необходимо определить суммарные капитальные затраты на разработку
и внедрения программы по формуле:

\begin{table}[h]
  \caption{Технико-экономические показатели проекта}
  \label{table:econ:calculated_data}
  \begin{tabular}{| >{\raggedright}m{0.797\textwidth}
                  | >{\centering\arraybackslash}m{0.15\textwidth}|}
    \hline
    \centering Наименование показателя & Величина показателя \\

    \hline
    1. Трудоемкость решения задачи, ч & \num\valTrz \\

    \hline
    2. Периодичность решения задачи, раз/год & \num\valk \\

    \hline
    3. Годовые текущие затраты, связанные с решением задачи, \rub & \num\valZt \\

    \hline
    4. Отпускная цена программы, \rub & \num\valTSpr \\

    \hline
    5. Степень новизны программы & В \\

    \hline
    6. Группа сложности алгоритма & 2 \\

    \hline
    7. Прирост условной прибыли, \rub & \num\valPu \\

    \hline
    8. Годовой экономический эффект, \rub & \num\valEF \\

    \hline
    9. Срок возврата инвестиций, лет & \num\valTvRounded \\

    \hline
  \end{tabular}
\end{table}

% \clearpage
\removeEquantionBeforeSpace{}

\begin{equation}
  \label{eq:econ:Ko}
  \text{К}_{\text{о}} = \text{К}_{\text{з}} + \text{Ц}_{\text{пр}},
\end{equation}
\begin{explanationx}
  \item[где] $ \text{К}_\text{з} $ -- капитальные и приравненные к ним затраты;
  \item $ \text{Ц}_\text{пр} $ -- отпускная цена программного продукта.
\end{explanationx}

Капитальные и приравненные к ним затраты определяются по формуле:

\begin{equation}
  \label{eq:econ:Kz}
  \text{К}_{\text{з}} = \text{Ц}_\text{пк}^\text{б} \cdot \left( 1 - x \cdot
  \frac{\text{Н}_\text{пк}^\text{а}}{100} \right) \cdot \text{Т}_\text{з} \cdot
  \frac{k}{\text{Ф}_\text{пк}},
\end{equation}
\begin{explanationx}
  \item[где] $ \text{Ц}_\text{пк}^\text{б} $ -- балансовая стоимость комплекта вычислительной техники,
  необходимой для решения задачи, \rub;
  \item $ x $ -- возраст используемого ПК, лет.
\end{explanationx}

Предположив, что для решения задачи будет использован ПК с возрастом
\insertNumSomethingText{\valx}{год}{года}{лет}, рассчитаем капитальные затраты:

\begin{equation}
  \label{eq:econ:valKz}
  \text{К}_{\text{з}} = \num\valTSpkb \cdot \left( 1 - \num\valx \cdot
  \frac{\num\valNpka}{100} \right) \cdot \num\valTz \cdot \frac{\num\valk}{\num\valFpk}
  \approx \num\valKz
\end{equation}

Суммарные капитальные затраты составят:

\begin{equation}
  \label{eq:econ:valKo}
  \text{К}_{\text{о}} = \num\valKz + \num\valTSpr = \num\valKo
\end{equation}

Годовой экономический эффект от замены одного программного продукта на аналогичный
при обработке информации определяется по формуле:

\begin{equation}
  \label{eq:econ:EF}
  \text{ЭФ} = \text{П}_\text{у} - E \cdot \text{К}_\text{о},
\end{equation}
\begin{explanationx}
  \item[где] $ E $ -- коэффициент эффективности, равный ставке по кредитам на рынке
  долгосрочных кредитов.
\end{explanationx}

Приняв $ E = \num\valE $, получим значение годового экономического эффекта:

\begin{equation}
  \label{eq:econ:valEF}
  \text{ЭФ} = \num\valPu - \num\valE \cdot \num\valKo \approx \num\valEF
\end{equation}

Срок возврата инвестиций определяется по формуле:

\begin{equation}
  \label{eq:econ:Tv}
  \text{Т}_\text{в} = \frac{\text{К}_\text{о}}{\text{П}_\text{у}} =
  \frac{\num\valKo}{\num\valPu} \approx \num\valTvRounded
\end{equation}

Результаты расчета сводятся в таблице~\ref{table:econ:calculated_data}.

В результате технико-экономического обоснования инвестиций в
разработку программного продукта были получены следующие значения
показателей их эффективности:
\begin{enumerate_num}
  \item Прирост условной прибыли составит \num\valPu { }\rub.
  \item Годовой экономический эффект составит \num\valEF { }\rub.
  \item Все инвестиции окупаются за \insertNumSomethingText{\valTvRounded}{год}{года}{лет}.
\end{enumerate_num}

Таким образом, разработка данного программного продукта является
эффективной  с учетом того, что он
разрабатывается как для нужд самого предприятия, так и для продажи сторонним компаниям.
Инвестиции в его реализацию целесообразны.