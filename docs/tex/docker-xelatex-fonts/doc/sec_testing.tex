\pagebreak
\section{ПРОГРАММА И МЕТОДИКА ИСПЫТАНИЙ}
\label{sec:testing}

В данном разделе будет рассмотрено тестирование разработанного программного обеспечения.

Так как данный продукт представляет из себя систему компонентов, которая использует полноценный кластер, то и тестирование будет производиться в двух напрвлениях:
\begin{itemize}
    \item юнит-тестирование;
    \item интеграционное тестирование.
\end{itemize}

Большое внимание будет уделяться именно интеграционному тестированию, так как оно позволяет проверить рабатоспособность всей системы.
Можно сказать, что такое тестирование определяет наличие ошибки в приложении.
Юнит-тестирование в свою очередь помогает определить, в каком именно модуле произошла ошибка.

Важность тестирования возрастает и с тем, что проект является открытым, и поэтому обязательно необходимо тестировать все изменения.
Также для удобства тестирования были использованы инструменты для непрерывной интеграции.
В частности для непрерывной интеграции использовались:
\begin{itemize}
    \item Jenkins;
    \item Gitlab - pipelines.
\end{itemize}


Jenkins представляет из себя Java веб приложение, которое запускается на хосте, и позволяет запускать команды на этом хосте.
Таким образом, с помощью файла \texttt{./Jenkinsfile} идёт описание инструкций, которые необходимо выполнить для репозитория на используемой машине.
Более подробно с Jankins можно ознакомится на сайте с его документацией~\cite{jenkins_documentation}.

Таким образом, если на используемом хосте установлен Docker, то это позволяет полностью развернуть всю необходимую инфраструктуру для запуска всей системы.
Также есть возможность настроить Jenkins на проверку удалённого репозитория по расписанию.
Если с момента последнего запуска, в удалённый репозиторий были закоммичены изменения, то как только jenkins проверит удалённый репозиторий - он выполнит команды, которые указаны в репозитории.
Такой режим работы называется трубой (англ. pipeline).
Также можно запускать команды и вручную.

Аналогичным инструментом является Gitlab - pipelines.
Благодаря возможности интеграции с сервисом Github - такой инструмент позволяет запускать 

\section{Юнит тестирование}

Для запуска тестов необходимо запустить скрипт \texttt{test.sh}.
С помощью данного скрипта вызывается докер, в котором разворачивается вся необходимая инфраструктура, а также и разработанные компоненты системы.
В качестве интеграционного теста происходит проверка отправки и получения сообщений, которые передаются с помощью kafka.

С помощью докера запускаются следующие компоненты:
\begin{itemize}
    \item zookeeper;
    \item kafka;
    \item отправитель.
\end{itemize}

Отправитель использует настройки для тестирования.
Однако, для взаимодействия со внешним сервисом необходим ключ, что уже было отмечено раньше.
По этой причине, ключ передаётся в качестве аргумента при запуске скрипта \texttt{test.sh}.
Скрипт подменяет значение по-умолчанию в файле конфигурации.
Таким образом, появляется возможность запускать интеграционные тесты с помощью скрипта.

Также, при использовании средств непрерывной интеграции появляется возможность задать такой ключ прямо в настройках пайплайна.
По этой причине, в качестве аргумента для скрипта \texttt{test.sh} в jenkinsfile используется переменная окружения:
\begin{lstlisting}
    sh "./test.sh --api-key ${OPEN_WEATHER_MAP_API_KEY}"
\end{lstlisting}

Данное значение устанавливается в самих настройках пайплайна в jenkins.
Такой подход позволяет скрыть действительно используемый API ключ от других участников проекта.
В свою очередь каждый участник сможет использовать свой собственный ключ для тестирования, либо для использования проекта.

После того, как все компоненты были запущены, отправитель перебирает установленный диапазон параметров, и пытается получить для них данные.
Всё действует точно также, как и при настоящей работе программы: сначала происходит проверка локального хранилища, а в случае отсутствия данных - происходит обращение к внешнему сервису.
Все отправленные сообщения отправляются в топик для тестирования.
Это позволяет при желании использовать тот же kafka брокер, который будет использоваться и при работе программы.
В таком случае, происходит тестирование продукта в <<полевых>> условиях.

После того, как сообщения были отправлены в брокер, отдельный получатель проверяет количество сообщений в тестовом топике.
Количество сообщений должно совпадать с диапазоном значений, который был использован при отправлении значений.
Если же количество сообщений не совпадает - значит какие-то данные не были отправлены в топик.
Это означает, что такая система не может полноценно работать, поэтому использование текущего окружения и настроек приведёт к ошибочному поведению программы.

Такое тестирование позволяет уловить само наличие ошибок, однако не предоставляет точной информации, в какой именно части произошёл сбой.
Поэтому, при тестировании ведётся логгирование программы, что позволяет отследить полный ход программы.
Возможные причины отсутствия сообщений:
\begin{itemize}
    \item недоступность kafka-брокера;
    \item используемый топик не был создан;
    \item был указан неверный путь к локальному хранилищу;
    \item был указан неверный API-ключ;
    \item используемый API-ключ не имеет необходимой подписки для использования API для получения индекса загрязнений;
    \item отсутствие связи с сервером;
    \item неработоспособность zookeeper сервера;
    \item используемый порт уже используется другим приложением;
    \item недоступный удалённый адрес распределённой системы при использовании таковой;
    \item системные ошибки.
\end{itemize}

К системным ошибкам относятся недостаток свободного места на диске, нехватка оперативной памяти, принудительное завершение процесса и тому подобное.
Остальные ошибки можно распознать при чтении log-файла.


\section{Юнит тестирование}

